\chapter{Implementacja} \label{chap:implementation}

Celem niniejszej pracy jest próba stworzenia takiego generatora.
Założenia funkcjonalne:

\begin{itemize}
 \item generator ma być na tyle elastyczny, aby móc obsłużyć wiele sposobów zdefiniowania dziedziny aplikacji,
 \item generator powinien pozwalać na wygenerowanie dowolnych plików tekstowych (skryptów SQL, kodu źródłowego, dokumentancji HTML itd.),
 \item generator nie musi generować logiki biznesowej aplikacji - wystarczy dziedzina
 \item wszystkie szablony generacji powinny być zdefiniowane w~ten sam sposób (w~tym samym języku)
\end{itemize}

Założenia niefunkcjonalne:

\begin{itemize}
 \item generator zostanie stworzony w~technolgii .NET Framework,
 \item ...
\end{itemize}

Szczególna uwaga zostanie poświęcona aplikacjom opartym o~architekturę CQRS i~wykorzystującym bazy danych typu NoSQL.
Specyficzną cechą takich aplikacji jest to, że operują one na modelach o~wysokim stopniu denormalizacji, co wiąże się z~masowo występującą duplikacją danych.

\emph{TODO: Opisać CQRS.}
\emph{TODO: Opisać Event Sourcing.}
\emph{TODO: Opisac NoSQL.}
\emph{TODO: Opisać rodzaje baz NoSQL.}
\emph{TODO: Wybrać bazę NoSQL i dlaczego Cassandra.}
\emph{TODO: Opisać Cassandrę.}
