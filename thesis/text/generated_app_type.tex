\chapter{Sprecyzowanie typu aplikacji generowanych przez narzędzie} \label{chap:generated_app_type}

Po zebraniu założeń dotyczących rdzenia narzędzia generującego aplikacje, nadszedł czas na wybór typu aplikacji, które będą generowane przez narzędzie.

%=======
\section{Wybór typu aplikacji}
%=======

\begin{itemize}
 \item wybrać typ aplikacji
 \item dlaczego CQRS
  \item bo zdenormalizowana dziedzina
  \item gdzie w CQRS zdefiniowana jest dziedzina (encje) aplikacji? Model do odczytywania nie zawiera przecież encji, a tylko widoki.
  \item model "read" i model "write" częściowo na siebie zachodzą (lub nawet "read" zawiera "write"). Jak uniknąć duplikacji metadanych?
\end{itemize}

Szczególna uwaga zostanie poświęcona aplikacjom opartym o~architekturę CQRS i~wykorzystującym bazy danych typu NoSQL.
Specyficzną cechą takich aplikacji jest to, że operują one na modelach o~wysokim stopniu denormalizacji, co wiąże się z~masowo występującą duplikacją metadanych.



%=======
\section{CQRS}
%=======

\begin{itemize}
 \item opisać CQRS
 \item że często idzie w parze z Event Sourcing
\end{itemize}



%=======
\section{Event sourcing}
%=======

\begin{itemize}
 \item opisać Event Sourcing
 \item że całość dobrze idzie w parze z NoSQL
\end{itemize}



%=======
\section{NoSQL}
%=======

\begin{itemize}
 \item opisać NoSQL
  \begin{itemize}
   \item opisać rodzaje baz NoSQL
   \item wybrać bazę NoSQL i dlaczego Cassandra
  \end{itemize}
\end{itemize}



%=======
\section{Cassandra}
%=======

opisać Cassandrę
