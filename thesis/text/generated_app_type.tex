\chapter{Sprecyzowanie typu generowanych aplikacji} \label{chap:generated_app_type}

Po zebraniu założeń dotyczących rdzenia narzędzia generującego aplikacje, nadszedł czas na wybór typu aplikacji, które będą generowane przez narzędzie.



%=======
\section{Wybór typu aplikacji}
%=======

Narzędzie powinno generować taki typ aplikacji, aby można było w~pełni zbadać jego użyteczność i~ocenić, na ile eliminuje ono duplikację.
Aby było to możliwe, narzędzie powinno generować aplikacje, w~których występuje dużo potencjalnych miejsc występowania duplikacji.

Wymaganie to spełniają aplikacje o~architekturze wielowarstwowej (ang. \emph{multi-tier architecture, n-tier architecture}~\cite{ntier}).


\subsection{Architektura wielowarstwowa}

Architektura wielowarstwowa to taka, w~której ogólne obszary przetwarzania danych w~aplikacji są fizycznie rozdzielone pomiędzy osobne komponenty.
Współpraca pomiędzy tymi komponentami jest zorganizowana w~taki sposób, że komponent $A$ może korzystać z~funkcjonalności komponentu $B$ tylko wtedy, gdy komponent $B$ należy do warstwy logicznie umiejscowionej nie wyżej niż warstwa, do której należy komponent $A$.

Przykładem, a jednocześnie najpopularniejszą realizacją tej architektury jest architektura trójwarstwowa (ang. \emph{three-tier architecture}), która wprowadza podział aplikcji na trzy warstwy:

\begin{enumerate}
 \item Warstwa prezentacji (ang. \emph{Presentation Layer}) - odpowiada za komunikację z~użytkownikiem aplikacji (np. poprzez interfejs graficzny) lub innymi systemami (np. poprzez usługi sieciowe); jest to warstwa logicznie najwyższa;
 \item Warstwa logiki biznesowej (ang. \emph{Business Logic Layer, BLL}) - odpowiada za przetwarzanie danych zgodnie z~wymaganiami funkcjonalnymi aplikacji;
 \item Warstwa dostępu do danych (ang. \emph{Data Access Layer, DAL}) - udostępnia mechanizmy odczytu i~zapisu danych składowanych przez aplikację (np. w~pamięci lub w~bazie danych); jest to warstwa logicznie najniższa.
\end{enumerate}

Współpracę pomiędzy warstwami architektury trójwarstwowej przedstawia diagram zamieszczony na rysunku~\ref{fig:three_tier}.

\begin{figure}[!ht]
 \begin{center}
  \scalebox{0.7}
  {
   \includegraphics{figures/generated_app_type/three_tier.png}
  }
 \end{center}
 \caption{Współpraca pomiędzy warstwami architektury trójwarstwowej~\cite{three_tier}.}
 \label{fig:three_tier}
\end{figure}


W~takiej architekturze elementy dziedziny aplikacji często mają swoje odwzorowanie w~każdej z~warstw, na przykład:

\begin{itemize}
 \item jako obiekty modelu w~warstwie dostępu do danych;
 \item jako obiekty biznesowe (ang. \emph{business object})~\cite{business_object} w~warstwie logiki biznesowej;
 \item jako modele widoków (ang. \emph{view model})~\cite{view_model} interfejsu użytkownika lub obiekty tranportu danych (ang. \emph{Data Transfer Object, DTO})~\cite{dto} usług sieciowych w~warstwie prezentacji.
\end{itemize}

To sprawia, że aplikacja o~architekturze wielowarstwowej jest narażona na powszechne występowanie duplikacji wiedzy na temat dziedziny aplikacji, a~tym samym dobrze nadaje się jako typ aplikacji generowanych przez narzędzie.



%=======
\section{CQRS}
%=======

Przypadkiem szczególnym architektury wielowarstwowej jest architektura CQRS (\emph{Command Query Responsibility Segregation})~\cite{cqrs_journey}.
Zakłada ona podział wszystkich działań w~aplikacji na dwa rodzaje:

\begin{itemize}
 \item zapytanie (ang. \emph{query}) - działanie polegające na pobraniu danych z~bazy danych (lub innego źródła danych);
 \item komenda (ang. \emph{command}) - działanie polegające na modyfikacji danych w~bazie danych.
\end{itemize}

Działania te w~architekturze CQRS są rozłączne.
Ich wykonywaniem zajmują się dwa osobne modele danych aplikacji:

\begin{itemize}
 \item model zapytań (ang. \emph{Query Model}) - model przeznaczony do odczytu danych;
 \item model komend (ang. \emph{Command Model}) - model przeznaczony do modyfikacji danych.
\end{itemize}

Modele te mogą być całkowicie rozłączne lub częściowo na siebie zachodzić.
Koncepcyjny schemat tej architektury przedstawia rysunek~\ref{fig:cqrs}.

\begin{figure}[!ht]
 \begin{center}
  \scalebox{0.5}
  {
   \includegraphics{figures/generated_app_type/cqrs.png}
  }
 \end{center}
 \caption{Schemat architektury CQRS~\cite{cqrs_fowler}.}
 \label{fig:cqrs}
\end{figure}


Podział odpowiedzialności pomiędzy komponenty przedstawia się następująco:

\begin{itemize}
 \item model zapytań zajmuje się odczytywaniem danych z~bazy danych;
 \item odpowiedzialnością modelu komend jest realizacja logiki biznesowej aplikacji, w~tym weryfikacja poprawności danych, aktualizacja danych w~bazie danych itd.;
 \item warstwa prezentacji (\emph{UI}):
  \begin{itemize}
   \item wyświetla dane pobrane z~modelu zapytań za pośrednictwem interfejsów (\emph{Service Interfaces}),
   \item przekazuje - w~postaci komend - akcje wykonywane przez użytkownika do modelu komend.
  \end{itemize}
\end{itemize}

Wprowadzenie podziału pomiędzy zapytanie i~komendę niesie ze sobą dwie ważne zalety:

\begin{itemize}
 \item skomplikowana dziedzina aplikacji może być podzielona na dwie prostsze dziedziny, co ułatwia jej zrozumiene i~operowanie na niej;
 \item zapytania i~komendy mogą być wykonywane równolegle, co poprawia wydajność aplikacji;
 \item zapytania są wykonywane na specjalnie przygotowanych dla nich danych (np. zmaterializowanych widokach bazy danych), co ma bardzo pozytywny wpływ na ich wydajność.
\end{itemize}

W~parze z~zaletami idą jednak wady:

\begin{itemize}
 \item synchronizacja obu modeli w~przypadku, gdy korzystają one z~osobnych źródeł danych może być kłopotliwa; problem ten nie występuje na przykład wtedy, gdy model komend operuje na tabelach bazy danych, a~model zapytań - na zmaterializowanych widokach, których źródłem danych są te tabele (synchronizacja modeli odbywa się wtedy automatycznie po stronie bazy danych);
 \item aby każde zapytanie mogło być obsłużone jak najszybciej, model jest w~dużym stopniu zdenormalizowany.
\end{itemize}

Konsekwencją drugiej wady jest to, że w~modelu zapytań masowo występuje duplikacja elementów dziedziny aplikacji.
Jest to dobry powód do tego, aby generator generował aplikacje oparte właśnie o~architekturę CQRS.
Dodatkowo, wybór tej architektury stworzy okazję do przyjrzenia się innym problemom związanym z~zastosowaniem architektury CQRS.
Te problemy to:

\begin{enumerate}
 \item Model komend i model zapytań częściowo na siebie zachodzą lub nawet model zapytań w~całości zawiera model komend - rodzi to dwa pytania:
 \begin{itemize}
  \item jak w~tej sytuacji uniknąć duplikacji wiedzy na temat dziedziny aplikacji?
  \item który model wybrać na “pojedynczą, jednoznaczną i~autorytatywną” (patrz: Zasada ``DRY'' w~rozdziale~\ref{chap:duplication}) reprezentację wiedzy o~dziedzinie aplikacji?
 \end{itemize}
 \item Model komend może nie być nigdzie fizycznie przechowywany - komendy mogą bezpośrednio aktualizować zdenormalizowaną strukturę tabel bazy danych.
 Gdzie w~takim przypadku należy umieścić wiedzę na temat encji należących do dziedziny apliacji?
 Model zapytań nie zawiera przecież encji dziedziny, a~tylko widoki na te encje.
\end{enumerate}

Architektura CQRS często idzie w~parze z~wykorzystaniem wzorca Event Sourcing i~baz danych typu NoSQL.
Zagadnienia te zostaną opisane w~kolejnych sekcjach.



%=======
\section{Event Sourcing}
%=======

Event Sourcing~\cite{eventSourcing_msdn} jest wzorcem architektonicznym, który realizuje przechowywanie stanu aplikacji poprzez przechowywanie wszystkich zdarzeń (ang. \emph{event}), które zaszły w~systemie od początku jego działania aż do stanu obecnego.
Terminem ``zdarzenie'' określa się tutaj dowolną akcję, która ma wpływ na stan systemu.


\subsection{Cechy zdarzenia}

Zdarzenie posiada następujące cechy~\cite{eventSourcing_intro}:

\begin{itemize}
 \item ma ono znaczenie dopiero wtedy, kiedy rzeczywiście wystąpi (np. ``Post został skomentowany'') - zdarzeń przyszłych nie bierze się pod uwagę;
 \item jest ono niezmienne - jako że zdarzenie zaszło w~przeszłości, nie może ono zostać zmienione ani cofnięte; jego skutki mogą jednak zostać zniwelowane przez inne zdarzenie (np. ``Komentarz został usunięty'');
 \item powinno ono mieć znaczenie biznesowe, a~nie implementacyjne - opisanie zajścia zdarzenia słowami: ``Do tabeli Comment dodano nowy rekord'' niesie ze sobą mniejszą wartość biznesową niż: ``Post został skomentowany''.
 \item zazwyczaj zawiera ono informacje na temat kontekstu, w~którym zaszło (np. ``Użytkownik $X$ dodał komentarz $Y$ pod postem $Z$'', ``Moderator $W$ usunął komentarz $Y$'');
 \item informacja o~zajściu zdarzenia jest informacją jednokierunkową od nadawacy (ang. \emph{publisher}) do odbiorcy (ang. \emph{subscriber}) lub wielu odbiorców; reakcja odbiorcy na zdarzenie nie jest bezpośrednio znana nadawcy.\\
\end{itemize}

Rejestrowanie zdarzeń odbywa się poprzez stworzenie obiektu opisującego daną akcję i~zapisanie go w~systemie.
Zdarzenia rejestrowane są w~dzienniku zdarzeń (ang. \emph{event log} lub \emph{event storage}).
Za reagowanie na zdarzenia, a~w~efekcie faktyczną zmianę stanu systemu odpowiadają wyznaczone do tego obiekty implementujące odpowiednie procedury obsługi (ang. \emph{event handlers}).
Obiekty te pobierają nieprzetworzone zdarzenia z~dziennika zdarzeń i~wykonują procedury.
Pojedyczne zdarzenie może zostać przetworzone przez wiele takich obiektów, wykonujących wiele osobnych modyfikacji stanu systemu.


\subsection{Zalety i~wady wzorca}

Wzorzec Event Sourcing wprowadza nastepujące zalety:

\begin{itemize}
 \item dziennik zdarzeń dostarcza informacji o~historii działań użytkoniwków systemu - informacje te mogą być przydatne podczas szukania przyczyn występienia błędów;
 \item aplikację można łatwo usunąć i~przywrócić do poprzedniego stanu poprzez zarejestrowanie wszystkich zdarzeń od nowa (np. podczas przenoszenia jej na inną maszynę);
 \item stan aplikacji można łatwo cofnąć do dowolnego punktu w~czasie - wystarczy wyczyścić stan aplikacji i~zarejestrować od nowa wszystkie zdarzenia, ktore zaszły przed wybranym punktem w~czasie (może to ułatwić szukanie przyczyn wystąpienia błędów);
 \item w~razie wykrycia zdarzenia będącego przyczyną wystąpienia błędu, można usunąć to zdarzenie z~dziennika i~zarejestrować od nowa wszystkie zdarzenia, które nastąpiły po nim; na podobnej zasadzie można skorygować kolejność zdarzeń zarejestowanych w~złej kolejności (z~tej zalety powinno się korzystać jedynie w~sytuacjach awaryjnych, ponieważ narusza ona cechę zdarzeń mówiącą o~ich niezmieności).
\end{itemize}

Wady:

\begin{itemize}
 \item implementacja dziennika zdarzeń nie jest łatwa - powinna ona gwarantować, że zdarzenia zostaną wpisane do dziennika w~kolejności ich zgłaszania (co nie jest oczywiste w~aplikacjach wielowątkowych);
 \item z~reguły reakcja na zdarzenie odbywa się asynchronicznie względem procesu zgłaszającego jego zajście - zmiany stanu systemu nie są więc widoczne natychmiast po zajściu zdarzenia; zjawisko to jest widoczne na przykład w~serwisie YouTube~\cite{youtube} - zdarzenie ``Obejrzano film'' jest przetwarzane znacznie wolniej niż zdarzenie ``Dodano komentarz'' (wyświetlenia filmu sprawdzane są pod kątem fałszywych wyświetleń mających na celu sztuczne zwiększenie popularności filmu~\cite{youtube:301}), co skutkuje wyświetleniem większej liczby komentarzy niż liczby wyświetleń pod danym filmem (jeśli komentarzy i~wyświetleń jest odpowiednio dużo);
 \item odtwarzanie stanu systemu, w~którym zarejestrowano tysiące zdarzeń może trwać bardzo długo (rozwiązaniem tego problemu jest tworzenie migawek (ang. \emph{snapshot}) systemu przechowujacych stan systemu co ileś zdarzeń);
 \item zmiany w~kodzie źródłowym systemu moga powodować, że dawno zarejestrowane zdarzenia przestaną być niekompatybilne z~nową wersją systemu.
\end{itemize}


\subsection{Przykłady aplikacji wykorzystujących Event Sourcing}

Aby łatwiej zrozumieć działanie wzorca, garść przykładów aplikacji używanych na co dzień, które są oparte o~Event Sorcing lub częściowo go wykorzystują:

\begin{itemize}
 \item systemy kontroli wersji - dziennik zdarzeń przechowuje zmiany dokonywane na plikach znajdujących się w~wersjonowanym katalogu;
 \item edytory tekstu i~edytory graficzne - dziennik zdarzeń przechowuje zmiany dokonywane na edytowanym pliku.
\end{itemize}

Wzorzec ten ma jednak zastosowanie nie tylko w~aplikacjach użytkowych.
W~połączeniu z~wzorcem CQRS, sprawdza się także w~aplikacjach klasy enterprise.


\subsection{Współpraca z~CQRS}

Zasady działania wzorca Event Sorcing w~systemie opartym o~architekturę CQRS są następujące~\cite{cqrs_es}:

\begin{itemize}
 \item rolę modelu komend pełni dziennik zdarzeń - jedynym zadaniem komend jest zarejestrowanie zdarzenia w~dzienniku;
 \item stan systemu jest przechowywany w~modelu zapytań;
 \item za synchronizację modelu zapytań z~modelem komend odpowiadają procedury obsługi zdarzeń (\emph{event handlers}) - synchronizacja ta odbywa się asynchronicznie;
 \item wzorzec Event Sourcing nie ma wpływu na warstwę prezentacji systemu.
\end{itemize}


Współpraca ta została schematycznie przedstawiona na rysunku~\ref{fig:cqrs_es}.

\begin{figure}[!ht]
 \begin{center}
  \scalebox{0.6}
  {
   \includegraphics{figures/generated_app_type/cqrs_es.png}
  }
 \end{center}
 \caption{Współpraca architetury CQRS z~wzorcem Event Sourcing~\cite{cqrs_info}.}
 \label{fig:cqrs_es}
\end{figure}



%=======
\section{NoSQL}
%=======

opisać NoSQL

\subsection{Dostępne bazy NoSQL}

opisać rodzaje baz NoSQL

\subsubsection{Wide Column Store}

\begin{itemize}
 \item Hadoop~\cite{hadoop} - platforma stworzona w~technologii Java, wykonująca obliczenia na dużych ilościach danych realizowane przez wiele rozproszonych aplikacji, służy przede wszstkim do przetwarzania danych, a~nie do ich składowania;
 \item Cassandra~\cite{cassandra} - stworzona w~technologii Java baza danych, której główne cele to skalowalność, wysoka wydajność i~brak pojedynczego punktu awarii (ang. \emph{Single Point of Failure}, \emph{SPOF});
 \item Amazon SimpleDB~\cite{simple_db} - komercyjna baza danych nastawiona na prostotę użytkowania i~zarządzania.
\end{itemize}

\subsubsection{Document Store}

\begin{itemize}
 \item MongoDB~\cite{mongo_db} - umożliwia wygodne przeszukiwanie danych i~tworzenie indeksów na dowolnych ich atrybutach, wspiera automatyczne dzielenie danych pomiędzy wiele maszyn (ang. \emph{sharding}); stworzona w~języku C++;
 \item CouchDB~\cite{couch_db} - stworzona w~języku Erlang baza danych udostępniająca przechowywane dane poprzez protokół HTTP; nastawiona na wykorzystanie w~aplikacjach webowych i~mobilnych;
 \item RavenDB~\cite{raven_db} - baza danych umożliwiająca wykonywanie dowolnych operacji (np. przebudowy struktury) bez potrzeby zatrzymywania korzystających z~niej aplikacji i~udostępniająca wygodny sposób przeszukiwania danych; oparta na platformie .NET;
\end{itemize}

\subsubsection{Key Value Store}

\begin{itemize}
 \item DynamoDB~\cite{dynamo_db}
 \item Azure Table Storage~\cite{azure_table_storage}
 \item Redis~\cite{redis}
\end{itemize}

\subsubsection{Graph Databases}

\begin{itemize}
 \item Neo4J~\cite{neo4j}
 \item TITAN~\cite{titan}
 \item Trinity~\cite{trinity}
\end{itemize}


\subsection{Dlaczego Cassandra}

wybrać bazę NoSQL i dlaczego Cassandra



%=======
\section{Cassandra}
%=======

opisać Cassandrę
