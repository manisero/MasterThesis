\chapter{Podsumowanie} \label{chap:outro}

Opisawszy dostępne metody redukcji duplikacji, jako preferowaną metodę Autor wybrał generację kodu źródłowego oraz innych artefaktów systemu.
Opracowane narzędzie zostało oparte o~tę właśnie metodę.

W~trakcie rozważań Autor doszedł do wniosku, że rodzajem duplikacji zazwyczaj przejawiającym się w~szerszym zakresie systemu niż inne rodzaje duplikacji jest duplikacja wiedzy na temat dziedziny aplikacji.
To jej zostało poświęcone najwięcej uwagi.



%=======
\section{Weryfikacja przydatności rozwiązania}
%=======

\subsection{generacja}

co dało się wygenerować, a czego nie
rozwiązanie powinno skutecznie wspomagać implementację dziedziny aplikacji,

\begin{itemize}
 \item ile pracy wymagałoby dodanie nowej funkcjonalności (API, eksport do arkusza) - przykład?
 \item jaką inną aplikację można wygenerować (przykład)
 \item jak łatwo wprowadza się zmiany w dziedzinie aplikacji
  \begin{itemize}
   \item problem: jak obsłużyć zmiany struktury bazy danych? (migracje)
  \end{itemize}
\end{itemize}

Generacji podlega znaczna część systemu, w~tym wszystkie pliki dotyczące bazy danych i~dokumentacji.
Ręcznie należy zaimplementować:

\begin{itemize}
 \item dostęp do danych przechowywanych w~bazie danych;
 \item procedury obsługi zdarzeń;
 \item mechanizm zgłaszania zdarzeń, zapisywania ich w~dzienniku zdarzeń i~wywoływania procedur ich obsługi;
 \item kod aplikacji webowej;
 \item kod HTML stron serwisu;
\end{itemize}


\subsection{opisanie, w~jaki sposób rozwiązanie wspomaga projetowanie i~implementację aplikacji}

rozszerzenie powinno być oparte o~wykorzystanie języka DSL
rozszerzenie powinno być przyjazne biznesowi (ang. \emph{business-friendly}),
business-friendly, DLL
można porównać proces modelowania UML, DSL i JSON
przedstawienie kroków postępowania wymaganych do zastosowania osiągnięgo rozwiązania; ref do sekcji w impl i dsl

\subsection{testowalność}

rozszerzenie powinno ułatwiać proces testowania aplikacji, w~której zostanie zastosowane,
ref do sekcji


\subsection{redukcja duplikacji}

jakiego rodzaju duplikacji udało się uniknąć


\subsection{elastyczność}

rozwiązanie powinno być elastyczne, tzn. jego poszczególne komponenty powinny być wymienne,
komponenty wymienne
można zastosować nie tylko do CQRS (ref do sekcji)
dsl nie elastyczny, ale taka jego natura


\subsection{wydajność}

rozwiązanie nie powinno mieć negatywnego wpływu na wydajność aplikaji, w~której zostanie zastosowane;


\subsection{ogółem}

ocena stopnia spełnienia ogólnych wymagań dotyczących rozwiązania;



%=======
\section{Wnioski}
%=======

Zaaplikowanie mechanizmu generacji artefaktów w~projekcie informatycznym redukuje duplikację, ułatwia utrzymywanie kodu, testowanie aplikacji.
Nie szkodzi nawet to że generuje się tylko część artefaktów.

Przeszkodą jest to, że ugenerycznić da się tylko część procesu generacji - sporo rzeczy jest specyficznych dla konkretnej aplikacji.
W każdym razie w większych projektach na pewno warto poświęcić trochę wysiłku, opracować DSL i zaimplementować mechanizm, bo potem to się opłaci - tym bardziej, im bardziej system się rozrośnie.

Przykład stworzonego DSL pokazuje, że w~proces projektowania można zaangażować osoby związane z~bisnesem, a~wynikiem wspólnej pracy może być dokument, do którego odnosić się będą programiści, testerzy i~biznes.
Dokument ten może stanowić bardzo ważny i~przydatny element dokumentacji systemu.



%=======
\section{Możliwości rozwoju rozwiązania}
%=======

Elastyczność rozwiązania sprawia, że możliwości jego rozwijania są szerokie.
ocena możliwości rozszerzenia rozwiązania; jest elastyczne więc spoko

Dwa kierunki rozwoju:


\subsection{Wzbogacenie funkcjonalności osiągnięgo rozwiązania}

Generacja scenariuszy testowych
Generacja elementów logiki biznesowej


\subsection{Integracja rozwiązania z~innymi narzędziami}

Alternatywnym kierunkiem rozwoju rozwiązania może być jego integracja z~innymi istniejącymi narzędziami wspomagającymi projektowanie lub implementację aplikacji opartych na architekturze CQRS lub korzystających z~bazy danych Cassandra.

Takie narzędzie powstało w~ramach pracy magisterskiej pt. ``Mechanizm modelowania danych i~mapowania obiektowego dla Apache Cassandry'' autorstwa mgr inż. Jakuba Turka, obronionej w~październiku 2014 roku.
Służy ono do modelowania i~generowania struktury rodzin kolumn bazy danych Cassandra na podstawie klas obiektów należących do dziedziny aplikacji.
Zapewnia także mechanizm mapownia pomiędzy obiektami a~danymi przechowywanymi w~bazie danych.

Integracja polegałaby na tym, że generator nie generowałby definicji rodzin kolumn, a~zamiast tego posługiwałby się wspomnianym mechanizmem modelowania.
Powstałe rozwiązanie zredukowałoby liczbę szablonów generacji używanych przez generator, a~także zwalniałoby jego użytkowników z~obowiązku implementacji mechanizmu dostępu do bazy danych.
Co więcej gwarantowałoby ono wysoką wydajność aplikacji w~obszarze komunikacji z~bazą danych.
Tym samym powstałoby narzędzie, które jedynie na podstawie zrozumiałego dla człowieka opisu aplikacji jest w~stanie wygenerować system o~złożonej, nierelacyjnej strukturze przechowywanych danych.
