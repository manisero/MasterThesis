\chapter{Podsumowanie} \label{chap:outro}

\begin{itemize}
 \item porównać proces modelowania UML, DSL i JSON
\end{itemize}



%=======
\section{Ocena rozwiązania}
%=======

\begin{itemize}
 \item co dało się wygenerować, a czego nie
 \item jakiej duplikacji udało się uniknąć
 \item ile pracy wymagałoby dodanie nowej funkcjonalności (API, eksport do arkusza) - przykład?
 \item jaką inną aplikację można wygenerować (przykład)
 \item jak łatwo wprowadza się zmiany w dziedzinie aplikacji
  \begin{itemize}
   \item problem: jak obsłużyć zmiany struktury bazy danych? (migracje)
  \end{itemize}
\end{itemize}

Generacji podlega znaczna część systemu, w~tym wszystkie pliki dotyczące bazy danych i~dokumentacji.
Ręcznie należy zaimplementować:

\begin{itemize}
 \item dostęp do danych przechowywanych w~bazie danych;
 \item procedury obsługi zdarzeń;
 \item mechanizm zgłaszania zdarzeń, zapisywania ich w~dzienniku zdarzeń i~wywoływania procedur ich obsługi;
 \item kod aplikacji webowej;
 \item kod HTML stron serwisu;
\end{itemize}


