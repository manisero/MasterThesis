\chapter{Podsumowanie} \label{chap:outro}

Opisawszy dostępne metody redukcji duplikacji, jako preferowaną metodę Autor wybrał generację kodu źródłowego oraz innych artefaktów systemu.
Opracowane narzędzie zostało oparte o~tę właśnie metodę.

W~trakcie rozważań Autor doszedł do wniosku, że rodzajem duplikacji zazwyczaj przejawiającym się w~szerszym zakresie systemu niż inne rodzaje duplikacji jest duplikacja wiedzy na temat dziedziny aplikacji.
To jej zostało poświęcone najwięcej uwagi.



%=======
\section{Ocena rozwiązania}
%=======

W~sekcji~\ref{sec:intro:scope} wymieniono wymagania dotyczące rozwiązania.
Wszystkie wymagania zostały spełnione:


\subsection{generacja}

co dało się wygenerować, a czego nie
rozwiązanie powinno skutecznie wspomagać implementację dziedziny aplikacji,

\begin{itemize}
 \item ile pracy wymagałoby dodanie nowej funkcjonalności (API, eksport do arkusza) - przykład?
 \item jaką inną aplikację można wygenerować (przykład)
 \item jak łatwo wprowadza się zmiany w dziedzinie aplikacji
  \begin{itemize}
   \item problem: jak obsłużyć zmiany struktury bazy danych? (migracje)
  \end{itemize}
\end{itemize}

Generacji podlega znaczna część systemu, w~tym wszystkie pliki dotyczące bazy danych i~dokumentacji.
Ręcznie należy zaimplementować:

\begin{itemize}
 \item dostęp do danych przechowywanych w~bazie danych;
 \item procedury obsługi zdarzeń;
 \item mechanizm zgłaszania zdarzeń, zapisywania ich w~dzienniku zdarzeń i~wywoływania procedur ich obsługi;
 \item kod aplikacji webowej;
 \item kod HTML stron serwisu;
\end{itemize}


\subsection{opisanie, w~jaki sposób rozwiązanie wspomaga projetowanie i~implementację aplikacji}

rozszerzenie powinno być oparte o~wykorzystanie języka DSL
rozszerzenie powinno być przyjazne biznesowi (ang. \emph{business-friendly}),
business-friendly, DLL
można porównać proces modelowania UML, DSL i JSON
przedstawienie kroków postępowania wymaganych do zastosowania osiągnięgo rozwiązania; ref do sekcji w impl i dsl


\subsection{Wpływ na testowalność wygenerowanej aplikacji}

Zastosowanie języka DSL jako sposobu opisu dziedziny systemu pozwala ułatwić testowanie generowanych aplikacji.
Ułatwienie to zostało opisane w~sekcji~\ref{sec:dsl:testing}.


\subsection{Redukcja duplikacji}

jakiego rodzaju duplikacji udało się uniknąć


\subsection{elastyczność}

rozwiązanie powinno być elastyczne, tzn. jego poszczególne komponenty powinny być wymienne,
komponenty wymienne
można zastosować nie tylko do CQRS (ref do sekcji)
dsl nie elastyczny, ale taka jego natura


\subsection{Wpływ na wydajność wygenerowanej aplikacji}

Jako że rozwiązanie zostało oparte na mechanizmie generacji kodu źródłowego, nie ma ono wpływu na wydajność aplikacji, w~której zostanie użyte.
Kod wygenerowany przez narzędzie, po skopilowaniu będzie działał tak samo jak kod napisany ręcznie.



%=======
\section{Możliwości rozwoju rozwiązania}
%=======

Elastyczność rozwiązania pozwala na jego łatwą modyfikację.
Z~tego względu możliwości jego rozwijania są szerokie.

Autor przewiduje dwa najbardziej obiecujące kierunki rozwoju:


\subsection{Wzbogacenie funkcjonalności}

Oczywistym sposobem ulepszenia rozwiązania jest wbogacenie go o~nowe funkcjonalności.
Sposób opisu dziedziny i~mechanizm generacji mogą zostać udoskonalone tak, aby generacji podlegała większa niż dotychczas część artefaktów systemu.

W~pierwszej kolejności zaimplementowane powinny być następujące funkcjonalności:

\begin{itemize}
 \item generacja scenariuszy testowych (patrz: sekcja~\ref{sec:dsl:testing}),
 \item generacja elementów logiki biznesowej (np. klas procedur obsługi zdarzeń, patrz: sekcja~\ref{sec:impl:summary}).
\end{itemize}


\subsection{Integracja rozwiązania z~innym narzędziem}

Alternatywnym kierunkiem rozwoju rozwiązania może być jego integracja z~innymi istniejącymi narzędziami wspomagającymi projektowanie lub implementację aplikacji opartych na architekturze CQRS lub korzystających z~bazy danych Cassandra.

Takie narzędzie powstało w~ramach pracy magisterskiej pt. ``Mechanizm modelowania danych i~mapowania obiektowego dla Apache Cassandry'' autorstwa mgr inż. Jakuba Turka, obronionej w~październiku 2014 roku.
Służy ono do modelowania i~generowania struktury rodzin kolumn bazy danych Cassandra na podstawie klas obiektów należących do dziedziny aplikacji.
Zapewnia także mechanizm mapownia pomiędzy obiektami a~danymi przechowywanymi w~bazie danych.

Integracja polegałaby na tym, że generator nie generowałby definicji rodzin kolumn, a~zamiast tego posługiwałby się wspomnianym mechanizmem modelowania.
Powstałe rozwiązanie zredukowałoby liczbę szablonów generacji używanych przez generator, a~także zwalniałoby jego użytkowników z~obowiązku implementacji mechanizmu dostępu do bazy danych.
Co więcej gwarantowałoby ono wysoką wydajność aplikacji w~obszarze komunikacji z~bazą danych.
Tym samym powstałoby narzędzie, które jedynie na podstawie zrozumiałego dla człowieka opisu aplikacji jest w~stanie wygenerować system o~złożonej, nierelacyjnej strukturze przechowywanych danych.



%=======
\section{Wnioski}
%=======

Przeszkodą uniemożliwiającą osiągniecie uiwersalnego generatora jest to, że wspólna dla wszystkich aplikacji jest jedynie część procesu generacji.
Wiele aspektów - takich jak sposób opisu dziedziny - jest specyficznych dla konkretnego systemu.

Mimo to, w~większych projektach z~pewnością warto jest poświęcić wysiłek potrzebny na opracowanie języka DSL i zaimplementowanie mechanizmu generacji.
W~dłuższej perspektywie okaże się to opłacalną decyzją - zaaplikowanie mechanizmu generacji artefaktów w~projekcie informatycznym pozwala zredukować duplikację, ułatwia utrzymywanie kodu i~testowanie aplikacji.
Korzyści te są obecne nawet w~sytuacji, w~której  generacji podlega tylko część artefaktów systemu.

Przykład stworzonego jęzka DSL pokazuje, że w~proces projektowania można zaangażować osoby związane z~bisnesem, a~wynikiem wspólnej pracy może być dokument, do którego odnosić się będą nie tylko programiści, lecz także testerzy i~pozostałe osoby związane z~projektem.
Dokument ten może stanowić bardzo ważny i~przydatny element dokumentacji systemu.