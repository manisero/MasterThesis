\chapter{Podsumowanie} \label{chap:outro}

Opisawszy dostępne metody redukcji duplikacji, jako preferowaną metodę Autor wybrał generację kodu źródłowego oraz innych artefaktów systemu.
Opracowane narzędzie zostało oparte o~tę właśnie metodę.

W~trakcie rozważań Autor doszedł do wniosku, że rodzajem duplikacji zazwyczaj przejawiającym się w~szerszym zakresie systemu niż inne rodzaje duplikacji jest duplikacja wiedzy na temat dziedziny aplikacji.
To jej zostało poświęcone najwięcej uwagi.



%=======
\section{Weryfikacja przydatności rozwiązania}
%=======

\subsection{generacja}

co dało się wygenerować, a czego nie
rozwiązanie powinno skutecznie wspomagać implementację dziedziny aplikacji,

\begin{itemize}
 \item ile pracy wymagałoby dodanie nowej funkcjonalności (API, eksport do arkusza) - przykład?
 \item jaką inną aplikację można wygenerować (przykład)
 \item jak łatwo wprowadza się zmiany w dziedzinie aplikacji
  \begin{itemize}
   \item problem: jak obsłużyć zmiany struktury bazy danych? (migracje)
  \end{itemize}
\end{itemize}

Generacji podlega znaczna część systemu, w~tym wszystkie pliki dotyczące bazy danych i~dokumentacji.
Ręcznie należy zaimplementować:

\begin{itemize}
 \item dostęp do danych przechowywanych w~bazie danych;
 \item procedury obsługi zdarzeń;
 \item mechanizm zgłaszania zdarzeń, zapisywania ich w~dzienniku zdarzeń i~wywoływania procedur ich obsługi;
 \item kod aplikacji webowej;
 \item kod HTML stron serwisu;
\end{itemize}


\subsection{opisanie, w~jaki sposób rozwiązanie wspomaga projetowanie i~implementację aplikacji}

rozszerzenie powinno być oparte o~wykorzystanie języka DSL
rozszerzenie powinno być przyjazne biznesowi (ang. \emph{business-friendly}),
business-friendly, DLL
można porównać proces modelowania UML, DSL i JSON
przedstawienie kroków postępowania wymaganych do zastosowania osiągnięgo rozwiązania; ref do sekcji w impl i dsl

\subsection{testowalność}

rozszerzenie powinno ułatwiać proces testowania aplikacji, w~której zostanie zastosowane,
ref do sekcji


\subsection{redukcja duplikacji}

jakiego rodzaju duplikacji udało się uniknąć


\subsection{elastyczność}

rozwiązanie powinno być elastyczne, tzn. jego poszczególne komponenty powinny być wymienne,
komponenty wymienne
można zastosować nie tylko do CQRS (ref do sekcji)
dsl nie elastyczny, ale taka jego natura


\subsection{wydajność}

rozwiązanie nie powinno mieć negatywnego wpływu na wydajność aplikaji, w~której zostanie zastosowane;


\subsection{ogółem}

ocena stopnia spełnienia ogólnych wymagań dotyczących rozwiązania;



%=======
\section{Możliwości rozszerzenia rozwiązania}
%=======

ocena możliwości rozszerzenia rozwiązania; jest elastyczne więc spoko
Generacja scenariuszy testowych
Generacja elementów logiki biznesowej
Integracja z~Turasem
