\chapter{Podsumowanie} \label{chap:outro}

Opisawszy dostępne metody redukcji duplikacji, jako preferowaną metodę Autor wybrał generację kodu źródłowego oraz innych artefaktów systemu.
Opracowane narzędzie zostało oparte o~tę właśnie metodę.

W~trakcie rozważań Autor doszedł do wniosku, że rodzajem duplikacji zazwyczaj przejawiającym się w~szerszym zakresie systemu niż inne rodzaje duplikacji jest duplikacja wiedzy na temat dziedziny aplikacji.
To jej zostało poświęcone najwięcej uwagi.



%=======
\section{Ocena rozwiązania}
%=======

W~sekcji~\ref{sec:intro:scope} wymieniono wymagania dotyczące rozwiązania.
Wszystkie wymagania zostały spełnione:


\subsection{Wsparcie procesu projektowania aplikacji i~przyjazność dla biznesu}

Rozwiązanie wspiera proces projektowania dziedziny aplikacji.
W~podstawowej wersji (patrz: rozdział~\ref{chap:implementation:single_type}) ułatwia projektantom i~programistom definicję i~utrzymywanie dziedziny aplikacji.

W~wersji rozszerzonej, korzystającej z~języka DSL (patrz: rozdział~\ref{chap:dsl}) pozwala zaangażować w~proces projektowania dziedziny osoby związane z~biznesem.
Ten sposób projektowania można porównać z~wykorzystaniem języka UML.
DSL ma jednak tę zaletę, że przypomina język naturalny, dzięki czemu jest bardziej zrozumiały dla osób niebędących programistami.

Kroki postępowania dla obu wersji przedstawiono w~sekcjach: \ref{sec:impl:summary} i~\ref{sec:dsl:summary}.


\subsection{Wsparcie implementacji aplikacji i~przykład zastosowania}

W~rozwiązaniu zastosowano mechanizm generacji, co przyśpiesza i~ułatwia proces implementacji wybranych komponentów aplikacji.
Rozwiązanie to jest szególnie skuteczne dla aplikacji opartych o~architekturę CQRS, ponieważ zwalnia programistów z~obowiązku niezależnego utrzymywania dwóch osobnych modeli: modelu komend i~modelu zapytań.

Oprócz klas obiektów należących do tych modeli, generacji podlega znaczna część komponentów systemu, w~tym wszystkie pliki dotyczące bazy danych i~dokumentacji.
Pełne zestawienie komponentów generowanych i~tych, które programiści powinni dostarczyć samodzielnie, znajduje się w~sekcjach \ref{sec:generated_artifacts} i~\ref{sec:impl:summary}.

Dowodem skuteczności generatora jest użycie go w~celu wygenerowania komponentów przykładowej aplikacji (patrz: sekcja~\ref{sec:sample_app}).


\subsection{Wpływ na testowalność wygenerowanej aplikacji}

Zastosowanie języka DSL jako sposobu opisu dziedziny systemu pozwala ułatwić testowanie generowanych aplikacji.
Ułatwienie to zostało opisane w~sekcji~\ref{sec:dsl:testing}.


\subsection{Redukcja duplikacji w~obrębie wygenerowanej aplikacji}

Opracowane narzędzie pozwala zredukować wszystkie rodzaje duplikacji opisane w~sekcji~\ref{sec:dupl_kinds}:

\subsubsection{Duplikacja czynności wykonywanych przez programistów}
Narzędzie jest w~stanie generować skrypty powłoki automatyzujące niektóre czynności często wykonywane przez programistów, takie jak inicjalizacja struktury bazy danych.

\subsubsection{Duplikacja kodu źródłowego aplikacji}
Narzędzie generuje wiele fragmentów kodu, które mają identyczną strukturę, np. klasy zdarzeń, czy interfejsy procedur obsługi zdarzeń.
Uwalnia to programistów od potrzebny powielania tych struktur.

\subsubsection{Duplikacja opisu dziedziny aplikacji}
Temu rodzajowi duplikacji zostało poświęcone najwięcej uwagi - narzędzie pozwala całkowicie go wyeliminować.
Autorytatywna wiedza na temat dziedziny aplikacji jest zawarta w~jednym miejscu: opisie systemu stanowiącym dane wejściowe generatora.


\subsection{Elastyczność generatora}

Podczas implementacji rozwiązania wielokrotnie kładziono nacisk na jego elastyczność.
Dzięki temu udało się osiągnąć wymienność kroków jego działania.

Dowodem elastyczności generatora jest powodzenie implementacji rozszerzenia rozwiązania, tak aby wykorzystywało język DSL (patrz: rozdział~\ref{chap:dsl}).
Implementacja ta wymagała modyfikacji jedynie dwóch kroków generacji aplikacji (sposobu opisu dziedziny i~algorytmu wyodrębnienia z~niego jednostek generacji).

Co więcej, rdzeń generatora może być użyty do generacji aplikacji innych niż oparte na architekturze CQRS (patrz: sekcja~\ref{sec:impl:summary}).


\subsection{Wpływ na wydajność wygenerowanej aplikacji}

Jako że rozwiązanie zostało oparte na mechanizmie generacji kodu źródłowego, nie ma ono wpływu na wydajność aplikacji, w~której zostanie użyte.
Kod wygenerowany przez narzędzie, po skopilowaniu będzie działał tak samo jak kod napisany ręcznie.



%=======
\section{Możliwości rozwoju rozwiązania}
%=======

Elastyczność rozwiązania pozwala na jego łatwą modyfikację.
Z~tego względu możliwości jego rozwijania są szerokie.

Autor przewiduje dwa najbardziej obiecujące kierunki rozwoju:


\subsection{Wzbogacenie funkcjonalności}

Oczywistym sposobem ulepszenia rozwiązania jest wbogacenie go o~nowe funkcjonalności.
Sposób opisu dziedziny i~mechanizm generacji mogą zostać udoskonalone tak, aby generacji podlegała większa niż dotychczas część artefaktów systemu.

W~pierwszej kolejności zaimplementowane powinny być następujące funkcjonalności:

\begin{itemize}
 \item generacja scenariuszy testowych (patrz: sekcja~\ref{sec:dsl:testing}),
 \item generacja elementów logiki biznesowej (np. klas procedur obsługi zdarzeń, patrz: sekcja~\ref{sec:impl:summary}).
\end{itemize}


\subsection{Integracja rozwiązania z~innym narzędziem}

Alternatywnym kierunkiem rozwoju rozwiązania może być jego integracja z~innymi istniejącymi narzędziami wspomagającymi projektowanie lub implementację aplikacji opartych na architekturze CQRS lub korzystających z~bazy danych Cassandra.

Takie narzędzie powstało w~ramach pracy magisterskiej pt. ``Mechanizm modelowania danych i~mapowania obiektowego dla Apache Cassandry'' autorstwa mgr inż. Jakuba Turka, obronionej w~październiku 2014 roku.
Służy ono do modelowania i~generowania struktury rodzin kolumn bazy danych Cassandra na podstawie klas obiektów należących do dziedziny aplikacji.
Zapewnia także mechanizm mapownia pomiędzy obiektami a~danymi przechowywanymi w~bazie danych.

Integracja polegałaby na tym, że generator nie generowałby definicji rodzin kolumn, a~zamiast tego posługiwałby się wspomnianym mechanizmem modelowania.
Powstałe rozwiązanie zredukowałoby liczbę szablonów generacji używanych przez generator, a~także zwalniałoby jego użytkowników z~obowiązku implementacji mechanizmu dostępu do bazy danych.
Co więcej gwarantowałoby ono wysoką wydajność aplikacji w~obszarze komunikacji z~bazą danych.
Tym samym powstałoby narzędzie, które jedynie na podstawie zrozumiałego dla człowieka opisu aplikacji jest w~stanie wygenerować system o~złożonej, nierelacyjnej strukturze przechowywanych danych.



%=======
\section{Wnioski}
%=======

Przeszkodą uniemożliwiającą osiągniecie uiwersalnego generatora jest to, że wspólna dla wszystkich aplikacji jest jedynie część procesu generacji.
Wiele aspektów - takich jak sposób opisu dziedziny - jest specyficznych dla konkretnego systemu.

Mimo to, w~większych projektach z~pewnością warto jest poświęcić wysiłek potrzebny na opracowanie języka DSL i zaimplementowanie mechanizmu generacji.
W~dłuższej perspektywie okaże się to opłacalną decyzją - zaaplikowanie mechanizmu generacji artefaktów w~projekcie informatycznym pozwala zredukować duplikację, ułatwia utrzymywanie kodu i~testowanie aplikacji.
Korzyści te są obecne nawet w~sytuacji, w~której  generacji podlega tylko część artefaktów systemu.

Przykład stworzonego jęzka DSL pokazuje, że w~proces projektowania można zaangażować osoby związane z~bisnesem, a~wynikiem wspólnej pracy może być dokument, do którego odnosić się będą nie tylko programiści, lecz także testerzy i~pozostałe osoby związane z~projektem.
Dokument ten może stanowić bardzo ważny i~przydatny element dokumentacji systemu.