\begin{figure}[!ht]
Szablon:

\begin{verbatim}
<#@ template language="C#" #>
<#@ parameter type="Schema.Model.View" name="Metadata" #>
<#@ parameter type="System.String" name="KeySpace" #>
USE "<#= KeySpace #>";

CREATE TABLE "<#= Metadata.Name #>" (
<# foreach (var field in Metadata.Fields) { #>
 "<#= field.Name #>" <#= field.Type #>,
<# } #>
 <#= CqlHelper.FormatPrimaryKey(Metadata) #>
);

<# foreach (var field in Metadata.Fields.Where(x => x.IsSearchable)) { #>
CREATE INDEX ON "<#= Metadata.Name #>" ("<#= field.Name #>");
<# } #>
\end{verbatim}

Wynik:

\begin{verbatim}
USE "Sample";

CREATE TABLE "Post" (
 "PostID" timeuuid,
 "Title" text,
 "Content" text,
 "CommentsNumber" int,
 "Author" text,
 PRIMARY KEY ("PostID")
);

CREATE INDEX ON "Post" ("Author");
\end{verbatim}

\caption{
 Szablon generacji definicji tabel i~przykładowy wynik jego działania.
 Metoda $FormatPrimaryKey$ klasy $CqlHelper$ formatuje kolumny klucza głównego.
}
\label{fig:single:template_table}
\end{figure}
