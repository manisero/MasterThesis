\begin{figure}[!ht]
\begin{center}
\begin{tikzpicture} 
\umlclass[x=5,y=5]{User}{ 
 \textbf{UserName}\\
 FirstName\\
 LastName
}{}

\umlclass[y=2]{Like}{
 \textbf{UserName}\\
 \textbf{PostID}\\
}{}

\umlclass[x=5]{Post}{ 
 \textbf{PostID}\\
 UserName\\
 Title\\
 Content
}{}

\umlclass[x=10,y=2]{Comment}{ 
 \textbf{CommentID}\\
 PostID\\
 UserName\\
 ParentCommentID\\
 Content
}{}

\umlassoc[mult1=1,mult2=*]{User}{Like}
\umlassoc[mult1=1,mult2=*]{Post}{Like}
\umlassoc[mult1=1,mult2=*]{User}{Comment}
\umlassoc[mult1=1,mult2=*]{Post}{Comment}
\umlassoc[mult1=1,mult2=*]{Comment}{Comment}
\umlassoc[mult1=1,mult2=*]{User}{Post}

\end{tikzpicture}
\end{center}

\caption{
 Schemat dziedziny przykładowej aplikacji.
 Pola wchodządze w~skład kluczy głównych zostały pogrubione.
}
\label{fig:single:domain_normalized}
\end{figure}
