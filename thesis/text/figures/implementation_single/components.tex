\begin{figure}[!ht]
\begin{center}
\begin{tikzpicture} 

\begin{umlcomponent}{Dotnet}
 \begin{umlcomponent}{Domain}
  \begin{umlcomponent}[x=0,y=2.3]{Views}
  \end{umlcomponent}
  \umlnote[x=-4.5, y=3, width=10em]{Views}{
   \begin{itemize}
    \item klasy widoków
   \end{itemize}
  }
  
  \begin{umlcomponent}[x=0,y=0]{Events}
  \end{umlcomponent}
  \umlnote[x=-4.5, y=0, width=10em]{Events}{
   \begin{itemize}
    \item klasy zdarzeń
   \end{itemize}
  }
 \end{umlcomponent}

 \begin{umlcomponent}[x=4,y=0]{DataAccess}
 \end{umlcomponent}
 \umlVHassemblyconnector[interface={\space}]{DataAccess}{Views}
 \umlnote[x=6.5, y=-3, width=10em]{DataAccess}{
  \begin{itemize}
   \item implementacja dostępu do bazy danych
  \end{itemize}
 }

 \begin{umlcomponent}[x=2,y=-4]{Logic}
 \end{umlcomponent}
 \umlVHVassemblyconnector[interface={\space}]{Logic}{Domain}
 \umlVHVassemblyconnector[interface={\space}]{Logic}{DataAccess}
 \umlnote[x=-4.5, y=-5, width=10em]{Logic}{
  \begin{itemize}
   \item interfejsy procedur obsługi zdarzeń
   \item klasy procedur obsługi zdarzeń
   \item implementacja dziennika zdarzeń
  \end{itemize}
 }
 
 \begin{umlcomponent}[x=2,y=-8]{WebSite}
 \end{umlcomponent}
 \umlVHVassemblyconnector[interface={\space}]{Logic}{WebSite}
 \umlnote[x=6.5, y=-7, width=10em]{WebSite}{
  \begin{itemize}
   \item kontrolery
   \item widoki
   \item modele widoków
  \end{itemize}
 }
\end{umlcomponent}

\begin{umlcomponent}[x=-4,y=-11]{CQL}
\end{umlcomponent}
\umlnote[x=-4, y=-14, width=10em]{CQL}{
 \begin{itemize}
  \item skrypty DML
  \item skrypty DQL
 \end{itemize}
}

\begin{umlcomponent}[x=1,y=-11]{Scripts}
\end{umlcomponent}
\umlHVHassemblyconnector[interface={\space}]{CQL}{Scripts}
\umlnote[x=1, y=-14, width=10em]{Scripts}{
 \begin{itemize}
  \item pomocnicze skrypty powłoki
 \end{itemize}
}

\begin{umlcomponent}[x=6,y=-11]{Documentation}
\end{umlcomponent}
\umlnote[x=6, y=-14, width=10em]{Documentation}{
 \begin{itemize}
  \item dokumentacja HTML
 \end{itemize}
}



\end{tikzpicture}
\end{center}
 
\caption{Diagram komponentów przykładowej aplikacji.}
\label{fig:single:components}
\end{figure}
