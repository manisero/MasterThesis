\begin{figure}[!ht]
Źródło generacji:

\dirtree{%
.1 database\_structure.
 .2 dbo.
  .3 tables.sql.
  .3 views.sql.
  .3 procedures.sql.
 .2 news.
  .3 tables.sql.
  .3 views.sql.
  .3 procedures.sql.
 .2 blogs.
  .3 tables.sql.
  .3 views.sql.
  .3 procedures.sql.
}

Wynik generacji:

\dirtree{%
.1 Model.
 .2 dbo.
  .3 User.cs.
  .3 ....
 .2 news.
  .3 News.cs.
  .3 Comment.cs.
  .3 ....
 .2 blogs.
  .3 Post.cs.
  .3 Comment.cs.
  .3 ....
}

\caption{Przykład organizacji plików przy generacji wielu katalogów wynikowych na podstawie wielu katalogów źródłowych (procedury składowane nie są podmiotem generacji).}
\label{fig:implementation_core:multipleFolders}
\end{figure}
