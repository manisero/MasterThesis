\chapter{Założenia dotyczące rdzenia narzędzia} \label{chap:implementation:core}

Przed przystąpieniem do prac nad generatorem aplikacji konkretnego typu, należy sformułować założenia dotyczące komponentu realizującego sam proces generacji artefaktów systemu, nazwanego rdzeniem narzędzia.



%=======
\section{Podstawowe założenia dotyczące rdzenia narzędzia} \label{sec:core:basic_requirements}
%=======

O ile elastyczność generatora konkretnego typu aplikacji może być ograniczona, o~tyle rdzeń powinien być na tyle uniwersalny, by mógł być użyty w~celu wygenerowania wielu rodzajów plików tekstowych, w~tym kodu źródłowego w~dowolnym języku, skryptów DLL, skryptów powłoki, dokumentancji w~formacie HTML lub XML itd.
Wszystkie rodzaje plików powinny być generowane w~ten sam sposób (np. na podstawie szablonów napisanych w~jednym języku, takim jak xslt~\cite{xslt}), tak aby programista korzystający z~komponentu nie musiał poznawać całej gamy języków lub narzędzi używanych do tworzenia szablonów generacji.
Pożądaną funkcjonalnością jest możliwość łatwiej wymiany domyślnie użwanego rodzaju szablonów na inny, tak aby użytkownik mógł w~łatwy sposób użyć w~nim szablonów stworzonych w~języku, ktory zna najlepiej.
Co więcej, rdzeń narzędzia nie pownien narzucać sposobu formatowania danych wejściowych (w~tym przypadku - opisu dziedziny aplikacji).

Dla wygody autora narzędzia zakłada się, że zarówno jego rdzeń jak i~generator aplikacji konkretnego typu zostanie stworzony w~technolgii .NET Framework



%=======
\section{Kroki generacji}
%=======

Generacja odbywać się będzie w~natępujących krokach:


\subsection{Wczytanie definicji dziedziny aplikacji}

Komponent na wejściu przyjmował będzie ścieżkę do pliku lub katalogu źródłowego (patrz: sekcja~\ref{sec:core:files_structure}).
Każdy plik katalogu źródłowego zostanie odwiedzony przez generator, a~jego treść zostanie zdeserializowana do obiektu odpowiadającej mu klasy.


\subsection{Zdeserializowanie definicji dziedziny aplkacji}

Jak wymieniono w~założeniach dotyczących rdzenia generatora (patrz: sekcja~\ref{sec:core:basic_requirements}), format opisu dziedziny aplikacji nie będzie narzucony z~góry.
Każdy plik źródłowy może być zapisany w~innym formacie (np. XML lub JSON).
Domyślnie wspierany będzie jeden format (patrz: sekcja~\ref{sec:core:domain_definition}), a~użytkownik generatora będzie mógł dla każdego pliku źródłowego określić wybrany i~dostarczony przez siebie sposób deserializacji.

Zdeserializowany obiekt dziedziny może posłużyć jako wskazanie na kolejne pliki reprezentujące elementy dziedziny.
Przykładem obrazującym potrzebę takiego działania może być sytuacja, w~której dany plik zawiera ogólne informacje na temat encji, a~pliki w~podkatalogu sąsiadującym z~tym plikiem zawierają definicje poszczególnych pól danej encji (patrz: sekcja~\ref{sec:core:files_structure:many_folders}).
Wtedy wczytanie i~deserializacja opisu pól encji odbędzie się dopiero po zdeserializowaniu opisu samej encji i~na tej podstawie określeniu, który katalog zawiera kolejne pliki opisujące jej pola. 


\subsection{Wyodrębnienie jednostek generacji}

Po uzyskaniu kompletnego obiektu - lub kolekcji obiektów - opisującego dziedzinę aplikacji, tzn. po zdeserializowaniu wszystkich plików zawierających dziedzinę aplikacji, generator będzie musiał przygotować kolejne obiekty, które będą podstawą do wygenerowania plików wynikowych.
Obiekty te nazwano jednostkami generacji.
Jednostką generacji może być na przykład opis pojedynczej encji dziedziny aplikacji lub, bardziej szczegółowo, pojedynczej tabeli bazy danych.

Aby zachować elastyczność, krok ten będzie musiał być zaimplementowany po stronie użytkownika komponentu, tzn. generatora aplikacji konkretnego typu.
Umożliwi to obsłużenie scenariusza, w~którym pojedynczy element opisu dziedziny aplikacji posłuży za źródło generacji wielu plików wynikowych (np. definicja tabeli bazy danych, definicja klasy w~kodzie źródłowym aplikacji i~kod HTML będący częścią dokumentancji systemu mogą być wygenerowane na podstawie pojedynczej jednostki generacji będącej opisem encji).
Odpowiedzialność wyodrębnienia jednostek generacji z~pełnego opisu dziedziny zostanie zrzucona na użytkownika komponentu dlatego, że rdzeń generatora nie jest w~stanie go zautomatyzować bez utraty elastyczności.
Aby jednak obsłużyć najprostsze scenariusze, pominięcie implementacji tego kroku w~aplikacji będącej użytkownikiem komponentu zaskutkuje potraktowaniem głównego obieku (lub kolekcji obiektów) dziedziny jako jednostki (lub osobnych jednostek) generacji.

Uzyskane jednostki generacji bezpośrednio posłużą do wygenerowania plików wynikowych. Pojedyncza jednostka będzie mogła odpowiadać jednemu lub wielu plikom wynikowym.


\subsection{Użycie szablonów do wygenerowania plików}

Ostatnim krokiem będzie użycie jednostek generacji w~celu wygenerowania plików wynikowych.
Domyślny mechanizm generacji pliku wynikowego oparty będzie o~wykorzystanie silnika generacji plików na postawie szablonu generacji (patrz: sekcja~\ref{sec:core:templating_engine}).

Rdzeń generatora przekaże daną jednostkę generacji odpowiedniemu szablonowi, a~wygenerowana treść zostanie umieszczona w~odpowiednim pliku.
Zadaniem użytkownika komponentu będzie dostarczenie zarówno szablonu generacji, jak i~ściezki, pod którą ma się znaleźć wygenerowany plik.

Aby zachować elastyczność, wykorzystywany silnik generacji będzie mógł zostać wymieniony na inny przez użytkonika generatora.



%=======
\section{Organizacja plików źródłowych i~wynikowych} \label{sec:core:files_structure}
%=======

Podstawową decyzją, którą należy podjąć w~trakcie precyzowania założeń dotyczących pierwszego kroku generacji jest organizacja i~format plików źródłowych, na których pracował będzie rdzeń narzędzia, a~także organizacja plików, które będzie w~stanie wygenerować.
Jak wspomniano wyżej, od rdzenia oczekuje się jak największej elastyczności - dlatego powinien on wspierać kilka scenariuszy.
Za przykład niech posłuży internetowy portal informacyjny.


\subsection{Pojedynczy plik źródłowy}

W~tym scenariuszu całość dziedziny aplikacji (lub innych informacji o~systmie) zawarta jest pojedynczym pliku.
Przykładem zastosowania może być pojedynczy skrypt SQL zawierający strukturę bazy danych używanej przez aplkację.
Taki plik, oprócz swojego standardowego przeznaczenia, tj. konfigurowania bazy danych, pełniłby rolę źródła generatora \emph{Code First}.
Na jego podstawie generowane byłyby pliki zawierające defnijcje klas będących częścią implementacji modelu dziedziny w~aplikacji.

Rysunek~\ref{fig:implementation_core:singleSourceFile} obrazuje przykładową organizację plików portalu.

\begin{figure}[!ht]
Źródło generacji:

\dirtree{%
.1 database\_schema.sql.
}

Wynik generacji:

\dirtree{%
.1 Model.
 .2 User.cs.
 .2 News.cs.
 .2 Comment.cs.
 .2 ....
}

\caption{}
\label{fig:implementation_core:singleSourceFile}
\end{figure}



\subsection{Pojedynczy katalog z~wieloma plikami źródłowymi}

Kontynuując przykład, w~miarę upływu czasu portal może rozrosnąć się na tyle, że wprowadzi możliwość prowadzenia blogów przez jego użytkowników.
Wtedy może wystąpić potrzeba podzielenia struktury bazy danych na kilka plików - np. według nazw schematów (ang. \emph{schema}), w~których znajdują się poszczególne tabele.
Wszystkie te pliki w~dalszym ciągu byłyby źródłem dla generatora, a~wynikowe klasy mogłyby być umieszczone w~osobnych katalogach (podzielonych według nazw schematow, w~których znajdują się odpowiadające im tabele).

Rysunek~\ref{fig:implementation_core:multipleSourceFiles} obrazuje przykład.

\begin{figure}[!ht]
Źródło generacji:

\dirtree{%
.1 database\_structure.
 .2 dbo.sql.
 .2 news.sql.
 .2 blogs.sql.
}

Wynik generacji:

\dirtree{%
.1 Model.
 .2 dbo.
  .3 User.cs.
  .3 ....
 .2 news.
  .3 News.cs.
  .3 Comment.cs.
  .3 ....
 .2 blogs.
  .3 Post.cs.
  .3 Comment.cs.
  .3 ....
}

\caption{Przykład organizacji plików przy generacji kilku katalogów wynikowych na podstawie wielu plików źródłowych (dbo - schemat wspólny, pozostałe - schematy właściwe obszarom, którymi zajmuje się portal).}
\label{fig:implementation_core:multipleSourceFiles}
\end{figure}



\subsection{Drzewo katalogów z~wielona plikami źródłowymi} \label{sec:core:files_structure:many_folders}

W~dłuższej perspektywie, w~opisywanym przykładzie może pojawić się potrzeba wprowadzenia podkatalogów dla poszczególnych schematów.
Pojedynczy podkatalog zawierałby osobne pliki zawierające definicje tabel, widoków i~procedur składowanych obecnych w~bazie danych.
Rdzeń narzędzia generującego powinien być w~stanie dotrzeć do wszystkich tych plików.
Przykład takiej organizacji został przedstawiony na rysunku~\ref{fig:implementation_core:multipleFolders}.

\begin{figure}[!ht]
Źródło generacji:

\dirtree{%
.1 database\_structure.
 .2 dbo.
  .3 tables.sql.
  .3 views.sql.
  .3 procedures.sql.
 .2 news.
  .3 tables.sql.
  .3 views.sql.
  .3 procedures.sql.
 .2 blogs.
  .3 tables.sql.
  .3 views.sql.
  .3 procedures.sql.
}

Wynik generacji:

\dirtree{%
.1 Model.
 .2 dbo.
  .3 User.cs.
  .3 ....
 .2 news.
  .3 News.cs.
  .3 Comment.cs.
  .3 ....
 .2 blogs.
  .3 Post.cs.
  .3 Comment.cs.
  .3 ....
}

\caption{Przykład organizacji plików przy generacji wielu katalogów wynikowych na podstawie wielu katalogów źródłowych (procedury składowane nie są podmiotem generacji).}
\label{fig:implementation_core:multipleFolders}
\end{figure}


Innym przykładem wykorzystującym ten scenariusz jest sytuacja, w~której pojedynczy plik wynikowy generowany jest na podstawie wielu plików źródłowych.
Przykładem może być zorganizowanie opisu dziedziny aplikacji w~taki sposób, aby informacje na temat każdego pola każdej encji umieszczone były w~osobnym pliku.
Rysunek~\ref{fig:implementation_core:multipleFolders_multipleInputFiles} obrazuje przykład.

\begin{figure}[!ht]
Źródło generacji:

\dirtree{%
.1 domain.
 .2 user.json.
 .2 user\_fields.
  .3 FirstName.json.
  .3 LastName.json.
  .3 ....
 .2 news.json.
 .2 news\_fields.
  .3 Title.json.
  .3 Content.json.
 .2 ....
}

Wynik generacji:

\dirtree{%
.1 Model.
 .2 User.cs.
 .2 News.cs.
 .2 ....
}

\caption{}
\label{fig:implementation_core:multipleFolders_multipleInputFiles}
\end{figure}


Taka organizacja może mieć zastosowanie w~przypadku, gdy w~systemie w~wielu miejscach występują jedynie fragmenty encji (np. w~licznych widokach bazy danych).
Wtedy każde pole może wymagać skonfigurowania dla niego miejsc, w~których występuje, czego skutkiem mogą być rozległe opisy każdego z~pól.
Opisy te najwygodniej byłoby przechowywać w~osobnych plikach.



%=======
\section{Sposób zdefiniowania dziedziny aplikacji} \label{sec:core:domain_definition}
%=======

O~ile założenie o~elastyczności generatora powinno dopuszczać zdefiniowanie dziedziny aplikacji w~dowolny sposób, używając dowolnego formatu plików zawierających opis poszczególnych elementów tej dziedziny, o~tyle generator powinien obsługiwać pewien domyślny sposób formatowania.
Poniżej przedstawiono kilka z~możliwych wyborów:

\subsection{UML}

Oczywistym wyborem sposobu opisu dziedziny aplikacji wydaje się być język UML~\cite{uml}, stworzony między innymi do tego właśnie celu.
Do opisu przykładowej dziedziny posłużyć może diagram klas przedstawiony na rysunku~\ref{fig:implementation_core:uml}.

\begin{figure}[!ht]
\begin{center}
 \begin{mpost}[use,mpsettings={input metauml; breadth=.667\mpdim{\linewidth};}]
  Class.HoundsViewModel("HoundsViewModel");
  Class.ConnectionManager("ConnectionManager")("+registerDataHandler()", "+unregisterDataHandler()", "+handleData()");
  Class.ChannelHandler("ChannelHandler");
  Class.IDataHandler("IDataHandler")("+handle()");
  Class.DataHandler("CheckpointEnteredInfoHandler")("+handle()");
  Class.CheckpointEnteredInfo("CheckpointEnteredInfo");
  
  classStereotypes.IDataHandler("<<interface>>");
  
  leftToRight(30)(ChannelHandler, ConnectionManager);
  leftToRight(40)(ConnectionManager, IDataHandler);
  leftToRight(40)(HoundsViewModel, DataHandler);
  topToBottom(30)(IDataHandler, DataHandler);
  topToBottom(30)(DataHandler, CheckpointEnteredInfo);
  
  drawObjects(HoundsViewModel, ConnectionManager, ChannelHandler, IDataHandler, DataHandler, CheckpointEnteredInfo);
  
  link(dependency)(ChannelHandler.e -- ConnectionManager.w);
  link(dependency)(HoundsViewModel.n -- ConnectionManager.s);
  link(aggregationUni)(IDataHandler.w -- ConnectionManager.e);
  link(inheritance)(DataHandler.n -- IDataHandler.s);
  link(associationUni)(HoundsViewModel.e -- DataHandler.w);
  link(associationUni)(DataHandler.s -- CheckpointEnteredInfo.n);
 \end{mpost}
\end{center}

\caption{Przykładowa dziedzina aplikacji przestawiona na diagramie klas będącym częścią języka UML.}
\label{fig:implementation_core:uml}
\end{figure}


Trzeba jednak zauważyć, ze taki opis nie jest wystarczająco elastyczny - posiada on zdefiniowany z~góry zestaw atrybutów.
Natomiast rzeczywiste scenariusze użycia generatora wymagać mogą atrybutów, których przewidzenie na etapie projektowania jego rdzenia jest niemożliwe.
Przykładowo, do opisu encji mogą należeć takie atrybuty jak:

\begin{itemize}
 \item opis encji, który powinien znaleźć się w~dokumentancji systemu;
 \item wersja systemu, w~której encja została wprowadzona;
 \item widoki bazy danych, kórych źródłem jest encja.
\end{itemize}

Diagram klas nie przewiduje przechowywania żadej z~tych informacji.

Co więcej, podstawową jednostką diagramu klas UML jest encja, co samo w~sobie jest ograniczeniem elastyczności.
Opis dziedziny aplikacji tworzony przez użytkownika generatora może natomiast za postawową jednostkę obierać na przykład pojedyncze pole encji (patrz: rysunek~\ref{fig:implementation_core:multipleFolders_multipleInputFiles}).
Do opisu pojedynczego pola encji, oprócz jego nazy i~typu, mogą należeć takie atrybuty, jak:

\begin{itemize}
 \item opis pola, który powinien znaleźć się w~dokumentancji systemu;
 \item wersja systemu, w~której pole zostało wprowadzone;
 \item widoki bazy danych, w~których występuje pole.
\end{itemize}

Diagram klas nie przewiduje przechowywania żadej z~tych informacji.


\subsection{XML}

Język XML~\cite{xml} jest powszechnie używany do opisu dziedziny.
Jest o niego oparty na przykład język WSDL (język definicji usług sieciowych, ang. \emph{Web Services Description Language})~\cite{wsdl} stosowany do opisu kontraktów (ang. \emph{contract}) realizowaych przez usługi sieciowe (ang. \emph{web service}).

XML jest pozbawiony wad języka UML - można w~nim zamodelować dowolne atrybuty.
Tę samą dziedzinę, która została przedstawiona na diagramie klas języka UML (rysunek~\ref{fig:implementation_core:uml}), ale wzbogaconą o~niedostępne na tym diagramie atrybuty, przedstawia rysunek~\ref{fig:implementation_core:xml}.

\begin{figure}[!ht]
\begin{verbatim}
<Entities>
 <Entity Name="User">
  <Description>The user of the system.</Description>
  <IntroducedIn>1.1</IntroducedIn>
  <Fields>
   <Field Name="FirstName">
    <Type>string</Type>
    <Description>The first name of the user.</Description>
   </Field>
   <Field Name="LastName">
    <Type>string</Type>
    <Description>The last name of the user.</Description>
   </Field>
  </Fields>
 </Entity>
 <Entity Name="News">
  <Description>The piece of news.</Description>
  <IntroducedIn>1.0</IntroducedIn>
  <Fields>
   <Field Name="Title">
    <Type>string</Type>
    <Description>The title of the piece of news.</Description>
   </Field>
   <Field Name="Content">
    <Type>string</Type>
    <Description>The content of the piece of news.</Description>
   </Field>
  </Fields>
 </Entity>
</Entities>
\end{verbatim}

\caption{Przykładowa dziedzina aplikacji opisana językiem XML.}
\label{fig:implementation_core:xml}
\end{figure}



\subsection{JSON}

Język JSON~\cite{json} posiada zestaw cech upodabniający go do języka XML.
Jest on jednak prostszy i~bardziej zwięzły - nie jest to język znaczników, więc nazwy atrybutów nie są duplikowane w~znaczniku otwierającym i~zamykającym.
JSON również jest szeroko stosowany do opisu dziedziny, na przykład do opisu metadanych usług sieciowych opartych o~protokół OData~\cite{odata}.

Rysunek~\ref{fig:implementation_core:json} przedstawia przykładowy opis dziedziny zapisany w~tym języku.

\begin{figure}[!ht]
\begin{verbatim}
[
 {
  "Name": "User",
  "Description": "The user of the system.",
  "IntroducedIn": "1.1"
  "Fields": [
   {
    "Name": "FirstName",
    "Type": "string",
    "Description": "The first name of the user.",
   },
   {
    "Name": "LastName",
    "Type": "string",
    "Description": "The last name of the user.",
   },
  ]
 },
 {
  "Name": "News",
  "Description": "The piece of news.",
  "IntroducedIn": "1.0",
  "Fields": [
   {
    "Name": "Title",
    "Type": "string",
    "Description": "The title of the piece of news.",
   },
   {
    "Name": "Content",
    "Type": "string",
    "Description": "The content of the piece of news.",
   },
  ]
 }
]
\end{verbatim}

\caption{Przykładowa dziedzina aplikacji zapisana w~języku JSON.}
\label{fig:implementation_core:json}
\end{figure}



\subsection{YAML}

Język YAML~\cite{yaml} jest pod względem użyteczności podobny do języków XML i~JSON.
Jest to język najbardziej zwięzły z~przedstawionych, jednak nie jest on używany w~żadnym popularnym standardzie.

Przykładowy opis dziedziny zapisany języku YAML został przedstawiony na rysunku~\ref{fig:implementation_core:json}.

\begin{figure}[!ht]
\begin{verbatim}
- Name:         User
  Description:  The user of the system.
  IntroducedIn: 1.1
  Fields:
   - Name:        FirstName
     Type:        string
     Description: The first name of the user.
     ...
     
   - Name:        LastName
     Type:        string
     Description: The last name of the user.
     ...
     
   ...
   
- Name:         News
  Description:  The piece of news.
  IntroducedIn: 1.0,
  Fields:
   - Name:        Title
     Type:        string
     Description: The title of the piece of news.
     ...
     
   - Name:        Content
     Type:        string
     Description: The content of the piece of news.
     ...
     
   ...
\end{verbatim}

\caption{Przykładowa dziedzina aplikacji opisana językiem YAML.}
\label{fig:implementation_core:yaml}
\end{figure}



\subsection{Domyślny język opisu dziedziny}

Jako że język UML nie spełnia wymagań rdzenia generatora w~zakresie elastyczności, domyślnie wspieranym językiem opisu dziedziny aplikacji będzie jeden z~pozostałych trzech opisanych wyżej.
Będzie to język JSON, za kórym przemawiają nastepujące fakty:

\begin{itemize}
 \item ma on najprostszą składnię za wszystkich trzech kandydatów;
 \item jest bardziej czytelny dla człowieka niż XML;
 \item jest szerzej znany i~stosowany niż YAML.
\end{itemize}

Aby zachować elastyczność, język opisu dziedziny będzie jednak można łatwo wymienić.



%=======
\section{Czy wymagać stworzenia schematu definicji dziedziny aplikacji?} \label{sec:core:domain_schema_requirement}
%=======

Kolejną decyzją jest wybór typu danych, do którego deserializoway będzie opis dziedziny aplikacji.
Wyboru należy dokonać pomiędzy dwoma przeciwstawnymi podejściami:

\begin{enumerate}
 \item Dane silnie typizowane, z~czym wiąże się wymóg określenia schematu opisu dziedziny aplikacji.
  Konsekwencje tego wyboru:
  \begin{itemize}
   \item wynikiem deserializacji opisu dziedziny aplikacji będzie obiekt konkretnego typu;
   \item wystąpienie w~opisie pola nieobecnego w~schemacie dziedziny zostanie zignorowane lub spowoduje błąd deserializacji.
  \end{itemize}
 \item Dane dynamiczne; brak wymogu istnienia schematu opisu dziedziny.
  Konsekwencje wyboru:
  \begin{itemize}
   \item wynikiem deserializacji opisu dziedziny będzie obiekt dynamiczny lub zbiór par klucz-wartość;
   \item wystąpienie w~opisie pola nieobecnego w~schemacie będzie akceptowalne - pole takie znajdzie się w~obiekcie powstałym w~wyniku deserializacji.
  \end{itemize}

\end{enumerate}

Pierwsza możliwość wydaje się bardziej korzystna
Przemawiają za nią następujące zalety:

\begin{itemize}
 \item generator będzie sprawdzał spójność opisu dziedziny aplikacji;
 \item ewentualne błędy opisu (na przykład tzw. literówki) zostaną wykryte na etapie deserializacji;
 \item szablony generacji będą mogły pracować na danych silnie typizowanych.
\end{itemize}

Domyślnie wspieranym podejściem będzie więc deserializacja opisu dziedziny do obiektu silnego typu.
Aby jednak zachować elastyczność, możliwa będzie deserializacja do typu dynamicznego.



%=======
\section{Wybór silnika generacji kodu} \label{sec:core:templating_engine}
%=======

Ostatnią decyzją w~ramach założeń dotyczących rdzenia generatora jest wybór silnika generacji kodu (ang. \emph{templating engine}) przez niego używanego.
Domyślnie używany silnik wybrano spośród następujących możliwości:


\subsection{XSLT}

XSLT (Xml Stylesheets Transformations)~\cite{xslt} jest językiem generacji dowolnych plików tekstowych na podstawie plików XML.
Jest to standard zaproponowany przez organizację W3C~\cite{w3c}.
Jego zaletą jest to, że szablony generacji tworzone są w~języku XML, co zwalnia programistę z~potrzeby poznawania kolejnego języka.
Jest on jednak mało czytelny, a~wprowadzanie zmian w~szablonie jest niewygodne.


\subsection{Razor}

Razor~\cite{razor} jest silnikiem generacji tekstu stworzonym na potrzeby platformy ASP.NET MVC.
Służy głównie do generacji kodu HTML, jednak może być używany w~celu tworzenia dowolnych plików tekstowych.

Wadą tego wyboru - niezależną od samego silnika - jest to, że szablony Razor są powszechnie używane w~typowych aplikacjach opartych o~platformę ASP.NET MVC, a~więc same mogą stanowić pliki wynikowe dla generatora.
Stworzenie szablonu Razor generującego inny szablon Razor jest możliwe, ale taki szablon byłby bardzo nieczytelny - co dyskwalifikuje ten silnik.


\subsection{T4}

T4~\cite{t4} to silnik generacji tekstu wbudowany w~środowisko programistyczne Microsoft Visual Studio.
Jest on przeznaczony do generowania plików tekstowych dowolnego typu na podstawie danych przekazanych szablonowi.

Szablon T4 nie jest interpretowany, a~kompilowany do kodu języka C\#, co daje następujące korzyści:

\begin{itemize}
 \item szybkość generacji tekstu jest większa niż w~przypadku pozostałych silników;
 \item szablon może odwoływać się do bibliotek zewnętrznych i~wywoływać ich metody (np. w~celu pobrania potrzebnych danych z~bazy danych);
 \item szablon może wywoływać inne szablony lub dziedziczyć po innym szablonie.
\end{itemize}

Taka integracja ze środowiskiem programistycznym i~platformą .NET niesie za sobą dalsze konsekwencje:

\begin{itemize}
 \item tworzenie szablonów jest ułatwione ze względu na podświetalnie i~podpowiadanie składni (zarówno języka szablonu i~jak korzystającego z~niego kodu C\#);
 \item silnik ten jest niedostępny dla programistów innych platform.
\end{itemize}


\subsection{Wybrany silnik}

Ze względu na fakt, że zarówno rdzeń generatora jak i~generator konkretnego typu aplikacji stworzone zostaną w~oparciu o~platformę .NET, silnikiem generacji domyślnie wspieranym przez rdzeń generatora będzie T4.
Powodem tego wyboru jest łatwość tworzenia jego szablonów w~środowisku Visual Studio.

Przykładowy szablon został przedstawiony na rysunku~\ref{fig:implementation_core:t4}.

\begin{figure}[!ht]
Szablon:

\begin{verbatim}
<#@ template language="C#" #>
<#@ parameter type="Sample.Schema.Entity" name="Entity" #>
namespace Sample
{
 /// <summary> <#= Entity.Description #> </summary>
 public class <#= Entity.Name #>
 {
<# foreach (var field in Entity.Fields) { #>
  /// <summary> <#= field.Description #> </summary>
  public <#= field.Type #> <#= field.Name #> { get; set; }
<# } #>
 }
}
\end{verbatim}

Wynik:

\begin{verbatim}
namespace Sample
{
 /// <summary> The user of the system. </summary>
 public class User
 {
  /// <summary> The first name of the User. </summary>
  public string FirstName { get; set; }
  
  /// <summary> The last name of the User. </summary>
  public string LastName { get; set; }
 }
}
\end{verbatim}

\caption{Przykładowy szablon T4 i~wynikowy kod C\#.}
\label{fig:implementation_core:t4}
\end{figure}


Podobnie jak w~przypadku pozostałych podjętych dezycji, używany przez generator silnik będzie mógł być zastąpiony przez inny.



%=======
\section{Podsumowanie}
%=======

Wymagania dotyczące rdzenia generatora zostały skompletowane.
Następnym krokiem jest sformułowanie założeń dotyczących generatora aplikacji konkretnego typu.
