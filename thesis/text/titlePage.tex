% title:
\begin{titlepage}

 \begin{tabular}{ll}
  \multirow{3}{*}{\includegraphics[scale=0.3]{figures/pw.jpg}} & POLITECHNIKA WARSZAWSKA                      \\
                                                               & Wydział Elektroniki i~Technik Informacyjnych \\
                                                               & Instytut Informatyki
 \end{tabular}
 
 \begin{flushright}
  Rok akademicki 2014/2015
 \end{flushright}

 \vspace{2cm}
 
 \begin{center}
  \LARGE PRACA DYPLOMOWA MAGISTERSKA
  
  \vspace{2cm}
  
  \large Michał Aniserowicz
  
  \vspace{2cm}
  
  \textbf{Generator aplikacji opartych o architekturę CQRS}
 \end{center}
 
 \vspace{3cm}
 
 \hfill Praca wykonana pod kierunkiem
 
 \hfill dra inż. Jakuba Koperwasa
 
 \vspace{3cm}

 \begin{flushleft}
  \begin{minipage}{7cm}
   Ocena: \dotfill \\ \\
   \hspace*{0cm} \dotfill \\[-0.7cm]
   \begin{center}
    \small\textit{Podpis Przewodniczącego Komisji Egzaminu Dyplomowego}
   \end{center}
  \end{minipage}
 \end{flushleft}

\end{titlepage}


% biography:
\newpage
\thispagestyle{empty}

\begin{flushright}
 Kierunek: Informatyka \\
 Specjalność: Inżynieria Systemów Informatycznych \\
 Data urodzenia: 1990.02.14 \\
 Data rozpoczęcia studiów: 2013.02.20 \\
\end{flushright}

\vspace*{3cm}

\begin{center}
 \textbf{\textbf{Życiorys}}
\end{center}

\vspace{1cm}
 
 Urodziłem się 14.02.1990 w Białymstoku.
 Wykształcenie podstawowe odebrałem w~latach 1997-2006 w~Publicznej Szkole Podstawowej nr~9 w~Białymstoku i~Publicznym Gimnazjum nr~2 im. 42 Pułku Piechoty w~Białymstoku.
 W~latach 2006-2009 uczęszczałem do III Liceum Ogólnokształcącego im. K. K. Baczyńskiego w~Białymstoku.
 W latach 2009-2013 studiowałem dziennie na Wydziale Elektroniki i~Technik Informacyjnych Politechniki Warszawskiej, na kierunku Informatyka.
 Od roku 2013 jestem studentem studiów dziennych drugiego stopnia na tym samym kierunku.
 W~2012 roku podjąłem pracę jako programista .NET w~firmie Fun and Mobile, gdzie po ok. roku pracy zostałem przywódcą zespołu.
 Obecnie kieruję kilkunastoosobowym działem .NET tej firmy.
 Moją pasją jest programowanie aplikacji w~technologii .NET Framework.

\vspace{2cm}

\begin{flushright}
 \begin{minipage}{5cm}
  \dotfill \\[-0.7cm]
  \begin{center}
  \small Podpis studenta
  \end{center}
 \end{minipage}
\end{flushright}

\vspace{4cm}

\begin{flushleft}
 Egzamin dyplomowy: \\
 Złożył egzamin dyplomowy w dniu: \dotfill \\
 z wynikiem: \dotfill \\
 Ogólny wynik studiów: \dotfill \\
 Dodatkowe uwagi i~wnioski Komisji: \dotfill \\
 \hspace{0cm} \dotfill
\end{flushleft}

 
% abstract:
\newpage
\thispagestyle{empty}

\textbf{Streszczenie} \\

Celem pracy jest stworzenie narzędzia wspomagającego projektowanie i~implementację aplikacji opartych o~architekturę CQRS.
Narzędzie powinno przyjmować opis dziedziny systemu, analizować go, i~na jego podstawie generować wybrane artefakty systemu.
Opis dziedziny powinien być na tyle zrozumiały dla człowieka, aby mógł być stworzony przez osobę niebędącą programistą.
Dodatkowymi wymaganiami dotyczącymi narzędzia są: redukcja duplikacji kodu źródłowego w~systemie oraz ułatwienie jego testowania.

Postawione cele udało się zrealizować - osiągnięte rozwiązanie pozwala w~wygodny sposób definiować i~utrzymywać dziedzinę systemu.
Do generowanych artekfaktów należą: kod źródłowy (C\#), skrypty definicji danych (Cassandra Query Language), skrypty powłowki inicjalizujące bazę danych oraz dokumentacja (HTML).

Treść pracy zawiera zbiór informacji teoretycznych dotyczących architektury CQRS, wzorca Event Sourcing oraz bazy danych Cassandra.
Opisuje także implementację generatora i~prezentuje jego działanie na przykładzie prostej aplikacji.


\vspace*{\stretch{1}}

\begin{center}
 \large \textbf{CQRS-architecture-based applications generator}
\end{center}

\vspace*{1cm}

\textbf{Summary} \\

The goal of this thesis is to create a~tool which would support the process of design and implementation of applications based on CQRS architecture.
The tool is required to take a description of a~domain of the system as an input and generate certain artifacts of the system as an output.
The description should be human readable, so that it can be created by people who are not programmers.
Additionally, the tool is required to reduce source code duplication within the generated system and to make the system easier to test.

All the goals have been fullfilled - achieved solution makes it convenient to define and maintain the domian of the generated application.
The generated artifacts are: source code (C\#), data definition scripts (Cassandra Query Language), shell scripts conducting database initialization, documentation (HTML).

The thesis contains theoretical information regarding CQRS architecture, Event Sourcing pattern and Cassandra database.
It also describes the implementation of the generator and presents its functionality by using it to generate a~simple application.


\vspace*{\stretch{1}}