\chapter{Sprecyzowanie problemu} \label{chap:the_problem}

Założenia funkcjonalne dotyczące core generatora:

\begin{itemize}
 \item generator ma być na tyle elastyczny, aby móc obsłużyć wiele sposobów zdefiniowania dziedziny aplikacji,
 \item generator powinien pozwalać na wygenerowanie dowolnych plików tekstowych (skryptów SQL, kodu źródłowego, dokumentancji HTML, skryptów powłoki itd.),
 \item generator nie musi generować logiki biznesowej aplikacji - wystarczy dziedzina
 \item wszystkie szablony generacji powinny być zdefiniowane w~ten sam sposób (w~tym samym języku)
\end{itemize}

Założenia niefunkcjonalne dotyczące core generatora:

\begin{itemize}
 \item generator zostanie stworzony w~technolgii .NET Framework,
 \item ...
\end{itemize}

Założenia dotyczące całości:

\begin{itemize}
 \item generator ma generować dziedzinę aplikacji CQRS wykorzystująej Cassandrę:
  \begin{itemize}
   \item schemat DLL
   \item klasy C\#: model Read, model Write
   \item dokumentancja HTML
  \end{itemize}
 \item ...
\end{itemize}

Szczególna uwaga zostanie poświęcona aplikacjom opartym o~architekturę CQRS i~wykorzystującym bazy danych typu NoSQL.
Specyficzną cechą takich aplikacji jest to, że operują one na modelach o~wysokim stopniu denormalizacji, co wiąże się z~masowo występującą duplikacją metadanych.
