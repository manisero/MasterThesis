\chapter{Wstęp} \label{chap:intro}

Architektura CQRS polega na wykorzystaniu dwóch osobnych modeli dziedziny opartego o~nią systemu: model zapisu danych i~model ich odczytu.
Pozwala to osiągnąć wysoką wydajność aplikacji, jednak sprawia pewne problemy.
Utrzymywanie dwóch modeli tej samej dziedziny powoduje trudności w~projektowaniu systemu, a~także wymusza istnienie duplikacji w~jego obrębie.
Celem niniejszej pracy magisterskiej jest opracowanie rozwiązania ułatwiającego projektowanie i~implementację aplikacji wykorzystujących tę architekturę.

Rozwiązanie przyjmie postać narzędzia mającego zastosowanie w~projektach programistycznych.
Podstawowym wymaganiem dotyczącym narzędzia jest zdolność wygenerowania znacznej części kodu źródłowego systemu, na potrzeby którego zostanie użyte.
Generacja powinna odbywać się na podstawie czytelnego dla człowieka opisu dziedziny systemu.
Składnia tego opisu powinna być na tyle łatwo zrozumiała dla człowieka, by w~jego tworzenie mogły zostać zaangażowane osoby niebędące programistami.
Dzięki temu, opis będzie mógł być sformułowany na etapie projektowania systemu i~stanowić dokument, do którego odnosić się będą zarówno programiści, jak i~osoby związane z~biznesem.

Wymagania drugorzędne stawiane generatorowi to zredukowanie duplikacji wiedzy występującej w~artefaktach systemu, a~także ułatwienie jego testowania.
Od narzędzia nie jest wymagana generacja kompletnego kodu źródłowego systemu ani~całkowita eliminacja występującej w~nim duplikacji.

Treść pracy zawiera podłoże teoretyczne, które stanowi podstawę dla implementacji rozwiązania.
Część praktyczna pracy obejmuje zaprojektowanie i~implementację generatora, a~także stworzenie przy jego pomocy przykładowej aplikacji.



%=======
\section{Zakres pracy} \label{sec:intro:scope}
%=======

Zakres niniejszej pracy obejmuje następujące działania:

\begin{enumerate}

 \item Przybliżenie rozwiązań stosowanych w~aplikacjach typu CQRS:
  \begin{itemize}
   \item omówienie architektury CQRS;
   \item omówienie wzorca Event Sourcing;
   \item omówienie ruchu NoSQL i~dostępnych baz danych tego typu.
  \end{itemize}

 \item Implementacja autorskiego rozwiązania wspierającego implementację aplikacji typu CQRS:
  \begin{itemize}
   \item ogólne wymagania dotyczące rozwiązania:
    \begin{itemize}
     \item rozwiązanie powinno skutecznie wspomagać implementację dziedziny aplikacji,
     \item rozwiązanie powinno być elastyczne, tzn. jego poszczególne komponenty powinny być wymienne,
     \item rozwiązanie nie powinno mieć negatywnego wpływu na wydajność aplikaji, w~której zostanie zastosowane;
    \end{itemize}
   \item sformułowanie szczegółowych założeń dotyczących rozwiązania;
   \item zaprojektowanie implementacji rozwiązania;
   \item wybór bazy NoSQL, która będzie wspierana przez rozwiązanie;
   \item sformułowanie przykładu aplikacji, która posłuży do zbadania przydatności rozwiązania;
   \item implementacja przykładowej aplikacji przy użyciu rozwiązania;
   \item ocena osiągniętej implementacji pod kątem spełnienia założeń.
  \end{itemize}
 
 \item Próba rozszerzenia rozwiązania dla jak największej przydatności w~przykładowej aplikacji:
  \begin{itemize}
   \item wymagania dotyczące rozszerzenia rozwiązania:
    \begin{itemize}
     \item rozszerzenie powinno być oparte o~wykorzystanie języka DSL,
     \item rozszerzenie powinno być przyjazne biznesowi (ang. \emph{business-friendly}),
     \item rozszerzenie powinno ułatwiać proces testowania aplikacji, w~której zostanie zastosowane,
     \item rozszerzenie nie musi być uniwersalne, tzn. może mieć zastosowanie tylko w~aplikacjach podobnych do przykładowej;
    \end{itemize}
   \item sformułowanie szczegółowych założeń dotyczących rozszerzenia rozwiązania;
   \item implementacja rozszerzenia rozwiązania;
   \item ocena osiągniętej implementacji pod kątem spełnienia założeń.
  \end{itemize}
 
 \item Weryfikacja przydatności rozwiązania:
  \begin{itemize}
   \item opisanie, w~jaki sposób rozwiązanie wspomaga projetowanie i~implementację aplikacji;
   \item ocena stopnia spełnienia ogólnych wymagań dotyczących rozwiązania;
   \item przedstawienie kroków postępowania wymaganych do zastosowania osiągnięgo rozwiązania;
   \item ocena możliwości rozszerzenia rozwiązania.
  \end{itemize}

\end{enumerate}

Oprócz wymagań wymienionych w~powyższym zakresie, pobocznym wymaganiem dotyczącym rozwiązania jest redukcja duplikacji występującej w~implementowanej aplikacji.
Z wymaganiem tym wiążą się następujące działania:

\begin{itemize}
 \item omówienie zjawiska duplikacji,
 \item przedstawienie rodzajów duplikacji,
 \item podanie przyczyn i~skutków występowania duplikacji w~obrębie aplikacji,
 \item opisanie możliwych metod redukcji duplikacji,
 \item wybór metody, która pozwoli na usunięcie jak największej liczby rodzajów duplikacji,
 \item włączenie wybranej metody do rozwiązania.
\end{itemize}



%=======
\section{Zawartość pracy}
%=======

Rozdział 2 (``Architektura CQRS'') zawiera przybliżenie pojęć i~wzorców stosowanych w~aplikacjach typu CQRS.
Sekcje 4.1-4.3 opisują architekturę CQRS i~wzorzec Event Sourcing.
Sekcje 4.4 i~4.5 poświęcone są bazom danych typu NoSQL.
Sekcja 4.4 wymienia popularne bazy tego typu i~podaje ich wspólne założenia.
Sekcja 4.5 przybliża zasady działania bazy Cassandra.

Rozdział 3 (``Duplikacja'') poświęcony jest zjawisku duplikacji.
Sekcje 2.1 i~2.2 zawierają jego opis, a~w~sekcjach 2.3-2.7 Autor podaje i~ocenia możliwe sposoby jego zwalczania.
W~sekcji 2.8 podjęty zostaje wybór sposobu zastosowanego w~docelowym rozwiązaniu, jak również zapada decyzja podzielenia rozwiązania na dwie części.

W~rozdziale 4 (``Założenia dotyczące rdzenia narzędzia'') znajduje się projekt pierwszej części rozwiązania - nazwanej rdzeniem generatora.
Sekcje 3.1 i~3.2 opisują podstawowe założenia i~koncepcję działania tworzonego narzędzia.
Sekcje 3.3-3.6 zawierają rozwiązania kolejnych problemów napotkanych podczas projektowania tej części rozwiązania.

Rozdział 5 (``Implementacja i ewaluacja generatora'') zawiera opis implementacji drugiej części rozwiązania - generatora aplikacji typu CQRS.
Sekcja 5.1 formułuje przykład aplikacji, która posłuży do zbadania przydatności rozwiązania.
Sekcja 5.2 wskazuje, które artefakty przykładowej aplikacji zostaną wygenerowane przez implementowane narzędzie.
Sekcje 5.3-5.5 zawierają opis kolejnych kroków implementacji generatora.
Sekcja 5.6 podsumowuje osiągniętą implementację generatora i~wskazuje, jak należy z~niego korzystać podczas projektowania aplikacji z~jego użyciem.

Rozdział 6 (``Wykorzystanie DSL'') został poświęcony rozszerzeniu generatora, opartemu o~język DSL.
Sekcje 6.1 i~6.2 zawierają opis składni opracowanego języka i~przykład jego zastosowania w~przykładowej aplikacji.
Sekcja 6.3 przestawia alogrytm przetwarzania opisu aplikacji zapisanego w~języku DSL.
Sekcja 6.4 wskazuję, w~jaki sposób rozszerzenie ułatwia testowanie wygenerowanej aplikacji.
W sekcji 6.5 podsumowano ostateczną implementację rozszerzenia i~wskazano, jak należy korzystać z~rozszerzonego generatora.

Rozdział 7 stanowi podsumowanie pracy - zawiera ocenę osiągniętego rozwiązania, a~także wskazuje perspektywy jego rozwoju.
