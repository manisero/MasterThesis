\chapter{Wstęp} \label{chap:intro}

Celem praktycznej części niniejszej pracy jest opracowanie narzędzia pozwalającego w~jak największym stopniu unikać duplikacji podrzas tworzenia aplikacji.
Narzędzie to będzie oparte o~mechanizmy generacji kodu źródłowego oraz innych artefaktów systemu.

Generacja została wybrała jako rozwiązanie problemu duplikacji dlatego, że pozwala zredukować duplikację nie tylko w~kodzie źródłowym aplikacji, a~w~obrębie całego systemu.
Co więcej, zastosowanie generacji u~podstawy systemu może zapobiec pojawianiu się duplikacji w~przyszłości - gdy pojawi się potrzeba stworzenia nowej funkcjonalności, nowego modułu aplikacji bądź nowego artefaktu systemu, programiści prawdopodobnie najpierw spróbują zaimplementować tę nową część systemu tak, aby była generowana na podstawie już istniejącej bazy.

Należy zaznaczyć, że od narzędzia będącego celem pracy nie jest wymagana całkowita eliminacja duplikacji w~systemie.
Główną postacią duplikacji, która będzie przedmiotem działania narzędzia jest duplikacja dziedziny aplikacji.
Wybór padł na tę właśnie postać dlatego, że przejawia się ona w~największym zakresie systemu.
Co więcej, użycie mechanizmów generacji nie przekreśla eliminacji innych postaci duplikacji.
Przykładowo, skrypty powłoki automatyzujące czynności wykonywane przez programistów również mogą być generowane lub mogą działać na wygenerowanych plikach - wtedy tym łatwiej będą się dostosowywać do zmian w~systemie.
Niektóre fragmenty kodu źródłowego logiki biznesowej lub testów jednostkowych aplikacji również mogą być generowane.
Jednak uwaga poświęcona zostanie głównie definicji dziedziny aplikacji.



%=======
\section{Zakres pracy}
%=======

\begin{enumerate}
 \item Omówienie zjawiska duplikacji:
  \begin{enumerate}
   \item przedstawienie rodzajów duplikacji,
   \item podanie przyczyn i~skutków występowania duplikacji w~kodzie źródłowym aplikacji,
   \item opisanie możliwych metod redukcji duplikacji,
   \item wybór metody redukcji duplikacji, która zostanie użyta w~implementacji autorskiego rozwiązania:
    \begin{itemize}
     \item wybór metody, która pozwoli na usunięcie jak największej liczby rodzajów duplikacji,
     \item omówienie wybranej metody: jej działania, możliwości i~dostępnych implementacji.
    \end{itemize}
  \end{enumerate}
 
 \item Implementacja autorskiego rozwiązania redukującego duplikację:
  \begin{itemize}
   \item ogólne wymagania dotyczące rozwiązania:
    \begin{itemize}
     \item rozwiązanie powinno uniwersalne, tzn. nadawać się do zastosowania w~aplikacjach różnych typów,
     \item rozwiązanie powinno być elastyczne, tzn. jego poszczególne komponenty powinny być wymienne,
     \item rozwiązanie nie powinno mieć negatywnego wpływu na wydajność aplikaji, w~której zostanie zastosowane,
    \end{itemize}
   \item sformułowanie szczegółowych założeń dotyczących rozwiązania,
   \item zaprojektowanie implementacji rozwiązania,
   \item sformułowanie przykładu aplikacji, która posłuży do zbadania przydatności rozwiązania,
   \item implementacja przykładowej aplikacji przy użyciu rozwiązania,
   \item ocena osiągniętej implementacji pod kątem spełnienia założeń.
  \end{itemize}
 
 \item Próba rozszerzenia rozwiązania dla jak największej przydatności w~przykładowej aplikacji:
  \begin{itemize}
   \item wymagania dotyczące rozszerzenia rozwiązania:
    \begin{itemize}
     \item rozszerzenie nie musi być uniwersalne, tzn. może mieć zastosowanie tylko w~aplikacjach podobnych do przykładowej,
     \item rozszerzenie powinno czynić rozwiązanie przyjazym biznesowi (ang. \emph{business-friendly}),
    \end{itemize}
   \item sformułowanie szczegółowych założeń dotyczących rozszerzenia rozwiązania,
   \item implementacja rozszerzenia rozwiązania,
   \item ocena osiągniętej implementacji pod kątem spełnienia założeń.
  \end{itemize}
  
 \item Weryfikacja przydatności rozwiązania:
  \begin{itemize}
   \item ocena stopnia, w~jakim rozwiązanie redukuje duplikację,
   \item ocena stopnia spełnienia ogólnych wymagań dotyczących rozwiązania,
   \item przedstawienie kroków postępowania dla osiągnięgo rozwiązania,
   \item porównanie tych kroków z~innym powszechnie stosowanym podejściem,
   \item ocena możliwości rozszerzenia rozwiązania.
  \end{itemize}

\end{enumerate}



%=======
\section{Zawartość pracy}
omówienie rozdziałów
%=======