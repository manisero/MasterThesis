\chapter{Wstęp} \label{chap:intro}

Celem niniejszej pracy magisterskiej jest opracowanie rozwiązania pozwalającego na zredukowanie duplikacji występującej w~kodzie źródłowym i~dokumentacji systemów informatycznych.
Rozwiązanie to przyjmie postać narzędzia mającego zastosowanie w~projektach programistycznych.

Podstawowym wymaganiem dotyczącym narzędzia jest jak największa redukcja duplikacji wybranego rodzaju.
Pozostałe główne wymagania to elastyczność i~możliwość zastosowania w~aplikacjach różnego typu.
Od narzędzia nie jest wymagana całkowita eliminacja duplikacji w~systemie.

Praca zawiera podłoże teoretyczne, które stanowi podstawę dla implementacji narzędzia.
Część praktyczna pracy obejmuje zaprojektowanie i~implementację narzędzia redukującego duplikację, a~także stworzenie przy jego pomocy przykładowej aplikacji.

W~trakcie prac nad narzędziem okazuje się, że jego stosowanie może mieć pozytywny wpływ na proces projektowania aplikacji tworzonych z~jego użyciem.
Aspekt ten zostaje zgłębiony, a~przeprowadzone rozważania pozwalają na osiągnięcie rozwiązania, w~którym w~łatwy sposób zwiększony zostaje udział osób niebędących programistami w~procesie projektowania dziedziny tworzonej aplikacji.
Rozwiązanie zostaje porównane z~podejściem szeroko stosowanym obecnie.



%=======
\section{Zakres pracy}
%=======

Zakres niniejszej pracy obejmuje następujące działania:

\begin{enumerate}

 \item Przybliżenie rozwiązań stosowanych w~aplikacjach typu CQRS:
  \begin{itemize}
   \item omówienie architektury CQRS;
   \item omówienie wzorca Event Sourcing;
   \item omówienie ruchu NoSQL i~dostępnych baz danych tego typu.
  \end{itemize}

 \item Implementacja autorskiego rozwiązania wspierającego implementację aplikacji typu CQRS:
  \begin{itemize}
   \item ogólne wymagania dotyczące rozwiązania:
    \begin{itemize}
     \item rozwiązanie powinno szczególnie skutecznie wspomagać implementację dziedziny aplikacji,
     \item rozwiązanie powinno być elastyczne, tzn. jego poszczególne komponenty powinny być wymienne,
     \item rozwiązanie nie powinno mieć negatywnego wpływu na wydajność aplikaji, w~której zostanie zastosowane;
    \end{itemize}
   \item sformułowanie szczegółowych założeń dotyczących rozwiązania;
   \item zaprojektowanie implementacji rozwiązania;
   \item wybór bazy NoSQL, która będzie wspierana przez rozwiązanie;
   \item sformułowanie przykładu aplikacji, która posłuży do zbadania przydatności rozwiązania;
   \item implementacja przykładowej aplikacji przy użyciu rozwiązania;
   \item ocena osiągniętej implementacji pod kątem spełnienia założeń.
  \end{itemize}
 
 \item Próba rozszerzenia rozwiązania dla jak największej przydatności w~przykładowej aplikacji:
  \begin{itemize}
   \item wymagania dotyczące rozszerzenia rozwiązania:
    \begin{itemize}
     \item rozszerzenie powinno być oparte o~wykorzystanie języka DSL,
     \item rozszerzenie powinno czynić rozwiązanie przyjaznym biznesowi (ang. \emph{business-friendly}),
     \item rozszerzenie powinno ułatwiać proces testowania aplikacji, w~której zostanie zastosowane,
     \item rozszerzenie nie musi być uniwersalne, tzn. może mieć zastosowanie tylko w~aplikacjach podobnych do przykładowej;
    \end{itemize}
   \item sformułowanie szczegółowych założeń dotyczących rozszerzenia rozwiązania;
   \item implementacja rozszerzenia rozwiązania;
   \item ocena osiągniętej implementacji pod kątem spełnienia założeń.
  \end{itemize}
 
 \item Weryfikacja przydatności rozwiązania:
  \begin{itemize}
   \item opisanie, w~jaki sposób rozwiązanie wspomaga projetowanie i~implementację aplikacji;
   \item ocena stopnia spełnienia ogólnych wymagań dotyczących rozwiązania;
   \item przedstawienie kroków postępowania wymaganych do zastosowania osiągnięgo rozwiązania;
   \item ocena możliwości rozszerzenia rozwiązania.
  \end{itemize}

\end{enumerate}

Oprócz wymagań wymienionych w~zakresie działań, pobocznym wymaganiem dotyczącym rozwiązania jest redukcja duplikacji kodu źródłowego implementowanej aplikacji.
Z wymaganiem tym wiążą się następujące działania:

\begin{itemize}
 \item omówienie zjawiska duplikacji,
 \item przedstawienie rodzajów duplikacji,
 \item podanie przyczyn i~skutków występowania duplikacji w~kodzie źródłowym aplikacji,
 \item opisanie możliwych metod redukcji duplikacji,
 \item wybór metody, która pozwoli na usunięcie jak największej liczby rodzajów duplikacji,
 \item włączenie wybranej metody do rozwiązania.
\end{itemize}



%=======
\section{Zawartość pracy}
%=======

Rozdział 2 (``Duplikacja'') poświęcony jest zjawisku duplikacji.
Sekcje 2.1 i~2.2 zawierają jego opis, a~w~sekcjach 2.3-2.7 Autor podaje i~ocenia możliwe sposoby jego zwalczania.
W~sekcji 2.8 podjęty zostaje wybór sposobu zastosowanego w~docelowym rozwiązaniu, jak również zapada decyzja podzielenia rozwiązania na dwie części.

W~rozdziale 3 (``Założenia dotyczące rdzenia narzędzia'') znajduje się projekt pierwszej części rozwiązania - nazwanej rdzeniem generatora.
Sekcje 3.1 i~3.2 opisują podstawowe założenia i~koncepcję działania tworzonego narzędzia.
Sekcje 3.3-3.6 zawierają rozwiązania kolejnych problemów napotkanych podczas projektowania tej części rozwiązania.

Rozdział 4 (``Sprecyzowanie typu generowanych aplikacji'') zawiera przybliżenie pojęć i~wzorców stosowanych w~aplikacjach typu CQRS.
Sekcje 4.1-4.3 opisują architekturę CQRS i~wzorzec Event Sourcing.
Sekcje 4.4 i~4.5 poświęcone są bazom danych typu NoSQL.
Sekcja 4.4 wymienia popularne bazy tego typu i~podaje ich wspólne założenia.
Sekcja 4.5 przybliża zasady działania bazy Cassandra.

Rozdział 5 (``Implementacja generatora'') zawiera opis implementacji drugiej części rozwiązania - generatora aplikacji typu CQRS.
Sekcja 5.1 formułuje przykład aplikacji, która posłuży do zbadania przydatności rozwiązania.
Sekcja 5.2 wskazuje, które artefakty przykładowej aplikacji zostaną wygenerowane przez implementowane narzędzie.
Sekcje 5.3-5.5 zawierają opis kolejnych kroków implementacji generatora.
Sekcja 5.6 podsumowuje osiągniętą implementację generatora i~wskazuje, jak należy z~niego korzystać podczas projektowania aplikacji z~jego użyciem.

Rozdział 6 (``Wykorzystanie DSL'') został poświęcony rozszerzeniu generatora, opartemu o~język DSL.
Sekcje 6.1 i~6.2 zawierają opis składni opracowanego języka i~przykład jego zastosowania w~przykładowej aplikacji.
Sekcja 6.3 przestawia alogrytm przetwarzania opisu aplikacji zapisanego w~języku DSL.
Sekcja 6.4 wskazuję, w~jaki sposób rozszerzenie ułatwia testowanie wygenerowanej aplikacji.
W sekcji 6.5 podsumowano ostateczną implementację rozszerzenia i~wskazano, jak należy korzystać z~rozszerzonego generatora.

Rozdział 7 stanowi podsumowanie pracy.
