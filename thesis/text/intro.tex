\chapter{Wstęp} \label{chap:intro}

Celem niniejszej pracy magisterskiej jest opracowanie rozwiązania pozwalającego na zredukowanie duplikacji występującej w~kodzie źródłowym i~dokumentacji systemów informatycznych.
Rozwiązanie to przyjmie postać narzędzia mającego zastosowanie w~projektach programistycznych.
Od narzędzia nie jest wymagana całkowita eliminacja duplikacji w~systemie.

Praca zawiera podłoże teoretyczne, które stanowi podstawę dla implementacji narzędzia.

Część praktyczna pracy obejmuje zaprojektowanie i~implementację narzędzia, a~także stworzenie przy jego pomocy przykładowej aplikacji.

W~trakcie prac nad narzędziem okazuje się, że jego stosowanie może mieć pozytywny wpływ na proces projektowania aplikacji tworzonych z~jego użyciem.
Aspekt ten zostaje zgłębiony, a~przeprowadzone rozważania pozwalają na osiągnięcie rozwiązania, w~którym w~łatwy sposób zwiększony zostaje udział osób niebędących programistami w~procesie projektowania dziedziny tworzonej aplikacji.



%=======
\section{Zakres pracy}
%=======

Zakres niniejszej pracy obejmuje następujące działania:

\begin{enumerate}
 \item Omówienie zjawiska duplikacji:
  \begin{enumerate}
   \item przedstawienie rodzajów duplikacji,
   \item podanie przyczyn i~skutków występowania duplikacji w~kodzie źródłowym aplikacji,
   \item opisanie możliwych metod redukcji duplikacji,
   \item wybór metody redukcji duplikacji, która zostanie użyta w~implementacji autorskiego rozwiązania:
    \begin{itemize}
     \item wybór metody, która pozwoli na usunięcie jak największej liczby rodzajów duplikacji,
     \item omówienie wybranej metody: jej działania, możliwości i~dostępnych implementacji.
    \end{itemize}
  \end{enumerate}
 
 \item Implementacja autorskiego rozwiązania redukującego duplikację:
  \begin{itemize}
   \item ogólne wymagania dotyczące rozwiązania:
    \begin{itemize}
     \item rozwiązanie powinno uniwersalne, tzn. nadawać się do zastosowania w~aplikacjach różnych typów,
     \item rozwiązanie powinno być elastyczne, tzn. jego poszczególne komponenty powinny być wymienne,
     \item rozwiązanie nie powinno mieć negatywnego wpływu na wydajność aplikaji, w~której zostanie zastosowane,
    \end{itemize}
   \item sformułowanie szczegółowych założeń dotyczących rozwiązania,
   \item zaprojektowanie implementacji rozwiązania,
   \item sformułowanie przykładu aplikacji, która posłuży do zbadania przydatności rozwiązania,
   \item implementacja przykładowej aplikacji przy użyciu rozwiązania,
   \item ocena osiągniętej implementacji pod kątem spełnienia założeń.
  \end{itemize}
 
 \item Próba rozszerzenia rozwiązania dla jak największej przydatności w~przykładowej aplikacji:
  \begin{itemize}
   \item wymagania dotyczące rozszerzenia rozwiązania:
    \begin{itemize}
     \item rozszerzenie nie musi być uniwersalne, tzn. może mieć zastosowanie tylko w~aplikacjach podobnych do przykładowej,
     \item rozszerzenie powinno czynić rozwiązanie przyjazym biznesowi (ang. \emph{business-friendly}),
    \end{itemize}
   \item sformułowanie szczegółowych założeń dotyczących rozszerzenia rozwiązania,
   \item implementacja rozszerzenia rozwiązania,
   \item ocena osiągniętej implementacji pod kątem spełnienia założeń.
  \end{itemize}
  
 \item Weryfikacja przydatności rozwiązania:
  \begin{itemize}
   \item ocena stopnia, w~jakim rozwiązanie redukuje duplikację,
   \item ocena stopnia spełnienia ogólnych wymagań dotyczących rozwiązania,
   \item przedstawienie kroków postępowania dla osiągnięgo rozwiązania,
   \item porównanie tych kroków z~innym powszechnie stosowanym podejściem,
   \item ocena możliwości rozszerzenia rozwiązania.
  \end{itemize}

\end{enumerate}



%=======
\section{Zawartość pracy}
omówienie rozdziałów
%=======