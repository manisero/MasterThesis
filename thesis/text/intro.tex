\chapter{Wstęp} \label{chap:intro}

\begin{itemize}
 \item Wstęp
 
 \item Duplikacja
 
 \item Sprecyzowanie problemu
  \begin{itemize}
   \item że zajmę się głównie dziedziną aplikacji
   \item że rozwiązanie będzie się składać z dwóch części:
    \begin{itemize}
     \item trzonu narzędzia do generacji:
     \item generatora aplikacji CQRS
    \end{itemize}
  \end{itemize}
  
 \item Trzon narzędzia do generacji:
  \begin{itemize}
   \item kroki generacji:
    \begin{itemize}
     \item wczytanie definicji dziedziny aplikacji
     \item zdeserializowanie jej
     \item wyodrębnienie konkretnych jednostek generacji (np. encji) (tego core nie zrobi)
     \item użycie zdefiniowanych (nie przez core) szablonów do wygenerowania plików na podstawie jednostek generacji
    \end{itemize}
   \item sposób zdefiniowania dziedziny aplikacji
    \begin{itemize}
     \item np. UML czy EMF (?) nie nadają się zbytnio bo działają na encjach, a chodzi o pojedyncze pola
     \item podkreślić, że podstawową jednostką nie jest encja, tylko pole
     \item format powinien być zupełnie dowolny
     \item domyślnie będzie JSON, ale można to łatwo podmienić
    \end{itemize}
   \item czy wymagać stworzenia schematu definicji dziedziny aplikacji?
    \begin{itemize}
     \item są dwie opcje:
      \begin{itemize}
       \item definicję dziedziny deserializować do dynamic, szablonom przekazywać dynamic (jest dowolność, ale nie wyłapie się błędów podczas deserializacji)
       \item definicję dziedziny deserializować do konkretnego typu, szablonom przekazywać konkretne typy
      \end{itemize}
     \item druga opcja wydaje się lepsza (opis dziedziny powinien być spójny)
     \item ale mechanizm będzie generyczny, decyzja będzie należała do konkretnego generatora
    \end{itemize}

   
   \item 
  \end{itemize}

\end{itemize}


\emph{TODO: Opisać CQRS.}
\emph{TODO: Opisać Event Sourcing.}
\emph{TODO: Opisac NoSQL.}
\emph{TODO: Opisać rodzaje baz NoSQL.}
\emph{TODO: Wybrać bazę NoSQL i dlaczego Cassandra.}
\emph{TODO: Opisać Cassandrę.}