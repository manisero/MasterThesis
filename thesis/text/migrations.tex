\chapter{Obsługa zmiany dziedziny aplikacji} \label{chap:migrations}

\begin{itemize}
 \item generator potrafi zrobić snapshot dziedziny / jednostki generacji
 \item snapshot będzie porównywany z obecnym stanem (lub innym snapshotem)
 
 \item co serializować w snapshocie (czyli na podstawie czego migrować):
  \begin{itemize}
   
   \item dziedzinę:
    \begin{itemize}
     \item zalety:
      \begin{itemize}
       \item odporne na zmiany schematu jednostek generacji
      \end{itemize}
     \item wady:
      \begin{itemize}
       \item i tak raczej trzeba będzie generować jednostki generacji
       \item do schematu dziedziny mogą tylko dochodzić nowe rzeczy (nic nie moze się zmienić / wypaść)
       \item trzeba i tak tworzyć jednostki generacji
       \item nie odporne na zmiany mechanizmu wyznaczania jednostek generacji na podstawie dziedziny (ten sam będzie używany do starych i nowych snapshotów)
      \end{itemize}
    \end{itemize}
   
   \item jednostki generacji
    \begin{itemize}
     \item zalety:
      \begin{itemize}
       \item efekty migracji i tak dotyczyć będą jednostek generacji, a nie dziedziny
       \item nie trzeba czytać dziedziny ze snapshota, tylko jednostki generacji (ale aktualną dziedzinę i tak trzeba czytać)
       \item można robić snapshot tylko niektórych jednostek generacji (np. tylko Views)
       \item odporne na zmiany schematu dziedziny
       \item zmiana jednostek generacji może wynikać z poprawki błędu działania mechanizmu generacji jednostek generacji - w tym przypadku taka poprawka też będzie potraktowana jako migracja
      \end{itemize}
     \item wady:
      \begin{itemize}
       \item do schematu jednostek generacji mogą tylko dochodzić nowe rzeczy (nic nie moze się zmienić / wypaść)
       \item na podstawie snapshota nie da się łatwo odtworzyć dziedziny (ale czy jest to potrzebne?)
      \end{itemize}
    \end{itemize}
   
  \end{itemize}
  
  
  \item implementacja dla:
   \begin{itemize}
    \item migracje bazy danych (cql)
    \item changelog w dokumentacji?
   \end{itemize}

\end{itemize}

