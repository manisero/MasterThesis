\chapter{Implementacja generatora} \label{chap:implementation:single_type}

\begin{itemize}
 \item Wybrór klienta Cassandry (?)  
\end{itemize}



%=======
\section{Co będzie generowane, a~co nie}
%=======

Założenia dotyczące całości:

\begin{itemize}
 \item generator ma generować dziedzinę aplikacji CQRS wykorzystująej Cassandrę:
  \begin{itemize}
   \item czy w ogóle generować schemat bazy danych (no schema) (?)
   \item schemat DLL (czyli tak)
   \item klasy C\#: model komend, model zapytań
   \item dokumentancja HTML
  \end{itemize}
 \item ...
\end{itemize}

Ogólnie nie chodzi o to, żeby w ogóle nie pisać nazw klas / właściwości - tylko o to, żeby nie trzeba było pamiętać o wszystkich miejscach, gdzie dana encja jest używana.



%=======
\section{Przykład generowanej aplikacji}
%=======

Sformułowanie przykładu aplikacji



%=======
\section{Sposób definicji dziedziny}
%=======

\begin{itemize}
 \item że będzie schemat opisu \ref{sec:core:domain_schema_requirement}
 \item ale że schema powinna być w jednym miejsu - będzie jako klasa w kodzie
 \item zdefiniowanie schematu opisu dziedziny na potrzeby CQRS (zapisać każdy w JSONie)
  \begin{itemize}
   \item każda tabela osobno (duplikacje)
   \item PresentIn
  \end{itemize}
 \item organizacja plików wejściowych i wyjściowych \ref{sec:core:files_structure}
\end{itemize}



%=======
\section{Podstawowe jednostki generacji}
%=======

\begin{itemize}
 \item entity
 \item event
 \item view
\end{itemize}



%=======
\section{Implementacja szablonów generacji}
%=======

Implementacja szablonów

\begin{itemize}
 \item CQL:
  \begin{itemize}
   \item create table
   \item select
  \end{itemize}
 \item C\#:
  \begin{itemize}
   \item entity
   \item event
   \item view
  \end{itemize}
 \item HTML (dokumentancja)
\end{itemize}



%=======
\section{Implementacja kolejnych modułów aplikacji}
%=======

\begin{itemize}
 \item DAL (repozytorium bazowe zahardkodowane, konkretnych nie da się wygenerować)
 \item BLL (obsługa zdarzeń zahardkodowana, da się wygenerować Eventy, a EventHandlery?)
 \item WEB (Nancy, da się wygenerować ViewModele, formy)
\end{itemize}
