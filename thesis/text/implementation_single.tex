\chapter{Implementacja generatora} \label{chap:implementation:single_type}

Generator aplikacji konkretnego typu będzie korzystał z~trzonu narzędzia w~celu wygenerowania części aplikacji opartej o~architekturę CQRS, korzystającej z~wzorca Event Souring i~z~bazy danych Cassandra.
Głównym celem generatora jest wyeliminowanie z~systemu duplikacji opisu dziedziny aplikacji.
Zostanie to osiągnięte przez umieszczenie pełnego opisu dziedziny w~jednym miejscu i na jego podstawie~generowanie innych artefaktów systemu, które duplikowałyby wiedzę na temat dziedziny.

W~celu przestawienia kolejnych kroków implementacji narzędzia, sformułowany zostanie przykład generowanej aplikacji.



%=======
\section{Przykład generowanej aplikacji}
%=======

Przykładowa aplikacja będzie serwisem internetowym, który umożliwia użytkownikom prowadzenie blogów.
Użytkownik serwisu (encja $User$) będzie mógł:

\begin{itemize}
 \item dodać wpis na swoim blogu (encja $Post$);
 \item polubić wpis innego użytkownika (encja $Like$);
 \item skomentować wpis innego użytkownika (encja $Comment$);
 \item skomentować komentarz innego użytkownika.
\end{itemize}

Dziedzina serwisu została przestawiona na rysunku~\ref{fig:single:domain_normalized}.
Dziedzina ta reprezentuje jednocześnie model komend systemu.

\begin{figure}[!ht]
\begin{center}
\begin{tikzpicture} 
\umlclass[x=4,y=5]{User}{ 
 UserName\\
 FirstName\\
 LastName
}{}

\umlclass[y=2]{Like}{ 
 UserName\\
 PostID
}{}

\umlclass[x=4]{Post}{ 
 PostID\\
 Author\\
 Title\\
 Content
}{}

\umlclass[x=9,y=2]{Comment}{ 
 CommentID\\
 PostID\\
 Author\\
 ParentCommentID\\
 Content
}{}

\umluniassoc{Like}{User}
\umluniassoc{Like}{Post}
\umluniassoc{Comment}{User}
\umluniassoc{Comment}{Post}
\umluniassoc{Comment}{Comment}
\umluniassoc{Post}{User}

\end{tikzpicture}
\end{center}

\caption{Znormalizowana dziedzina.}
\label{fig:single:domain_normalized}
\end{figure}


\ref{fig:single:domain_denormalized}
\begin{figure}[!ht]
\begin{center}
\begin{tikzpicture} 
\umlclass[y=4]{User}{ 
 \textbf{UserName}\\
 FirstName\\
 LastName
}{}

\umlclass{UserLike}{ 
 \textbf{UserName}\\
 \textbf{PostID}\\
 PostTitle
}{}

\umlclass[x=4.5,y=4]{Post}{ 
 \textbf{PostID}\\
 UserName\\
 Title\\
 Content\\
 CommentsNumber
}{}

\umlclass[x=4.5]{PostLike}{ 
 \textbf{PostID}\\
 \textbf{UserName}\\
 FirstName\\
 LastName
}{}

\umlclass[x=10,y=2]{Comment}{ 
 \textbf{CommentID}\\
 PostID\\
 UserName\\
 ParentCommentID\\
 Content
}{}

\end{tikzpicture}
\end{center}

\caption{Schemat modelu zapytań przykładowej aplikacji.}
\label{fig:single:domain_denormalized}
\end{figure}


Artefaktami generowanymi na podstawie opisu dziedziny aplikacji będą:

\begin{itemize}
 \item skrypty DLL definiujące strukturę bazy danych;
 \item klasy wchodzące w~skład modelu komend (patrz: sekcja~\ref{sec:cqrs});
 \item klasy wchodzące w~skład modelu zapytań (patrz: sekcja~\ref{sec:cqrs});
 \item klasy reprezentujące zdarzenia zachodzące w~systmie (patrz: sekcja~\ref{sec:event_sourcing});
 \item klasy reprezentujące dane wyświetlane w~interfejscie użytkownika (\emph{view models});
 \item dokumentację systemu w~formacie HTML;
 \item pomocnicze skrypty powłoki automatyzujące rutynowe czynności wykonywane przez programistów (patrz: sekcje~\ref{sec:tasks_dupl}, \ref{sec:tasks_scripts}).
\end{itemize}

We wszystkich tych elementach zazwyczaj powielana jest wiedza na temat dziedziny aplikacji.
Elementy mniej narażone na duplikację dziedziny, takie jak klasy realizujące logikę biznesową aplikacji, nie będą generowane przez narzędzie.



%=======
\section{Sposób definicji dziedziny}
%=======

\begin{itemize}
 \item że będzie schemat opisu \ref{sec:core:domain_schema_requirement}
 \item ale że schema powinna być w jednym miejsu - będzie jako klasa w kodzie
 \item zdefiniowanie schematu opisu dziedziny na potrzeby CQRS (zapisać każdy w JSONie)
  \begin{itemize}
   \item każda tabela osobno (duplikacje)
   \item PresentIn
  \end{itemize}
 \item które typy umieścić w definicji? CQLowe, .NETowe, czy własne? (nie ma potrzeby własnych, bo trzeba by robić dwie mapy; lepiej CQLowe bo jeden .NETowy mapuje się na wiele CQLowych) (że będzie potrzebna TypesMap.GetDotNetType)
 \item organizacja plików wejściowych i wyjściowych \ref{sec:core:files_structure}
\end{itemize}

\ref{fig:single:model_presentIn}
\begin{figure}[!ht]
\begin{verbatim}
{
 "Name": "User",
 "Fields": [
  {
   "Name": "UserName",
   "Type": "text",
   "IsKey": true,
   "PresentInViews": [
    { "Name": "Post" },
    { "Name": "PostLike", "IsKey": true },
    ...
   ],
   "PresentInEvents": [
    "PostPublishedEvent",
    ...
   ]
  },
  {
   "Name": "FirstName",
   "Type": "text",
   "PresentInViews": [ { "Name": "UserLike" } ]
  },
  {
   "Name": "LastName",
   "Type": "text",
   "PresentInViews": [ { "Name": "UserLike" } ]
  }
 ]
}
\end{verbatim}
\caption{Fragment opisu encji $User$ w~wykorzystaniem konstrukcji ``PresentIn''.}
\label{fig:single:model_presentIn}
\end{figure}


\ref{fig:single:model_perView}
\begin{figure}[!ht]
\begin{verbatim}
[
 {
  "Name": "User",
  "Fields": [
   { "Name": "UserName", "Type": "text", "IsKey": true },
   { "Name": "FirstName", "Type": "text" },
   { "Name": "LastName", "Type": "text" }
  ]
 },
 {
  "Name": "PostLike",
  "Fields": [
   ...
   { "Name": "UserName", "Type": "text", "IsKey": true },
   { "Name": "UserFirstName", "Type": "text" },
   { "Name": "UserLastName", "Type": "text" }
  ]
 },
 ...
]
\end{verbatim}
\caption{Fragment opisu dziedziny w~notacji ``per view''.}
\label{fig:single:model_perView}
\end{figure}



%=======
\section{Podstawowe jednostki generacji}
%=======

\begin{itemize}
 \item entity
 \item event
 \item view
\end{itemize}



%=======
\section{Implementacja szablonów generacji}
%=======

Implementacja szablonów

\begin{itemize}
 \item CQL:
  \begin{itemize}
   \item create table
   \item select
  \end{itemize}
 \item C\#:
  \begin{itemize}
   \item entity
   \item event
   \item view
  \end{itemize}
 \item HTML (dokumentancja)
\end{itemize}

\ref{fig:single:template_table}
\begin{figure}[!ht]
Szablon:

\begin{verbatim}
<#@ template language="C#" #>
<#@ parameter type="Schema.Model.View" name="Metadata" #>
CREATE TABLE "<#= Metadata.Name #>" (
<# foreach (var field in Metadata.Fields) { #>
 "<#= field.Name #>" <#= field.Type #>,
<# } #>
 <#= CqlHelper.FormatPrimaryKey(Metadata) #>
);

<# foreach (var field in Metadata.Fields.Where(x => x.IsSearchable)) { #>
CREATE INDEX ON "<#= Metadata.Name #>" ("<#= field.Name #>");
<# } #>
\end{verbatim}

Wynik dla widoku $Post$:

\begin{verbatim}
CREATE TABLE "Post" (
 "PostID" timeuuid,
 "Title" text,
 "Content" text,
 "CommentsNumber" int,
 "UserName" text,
 PRIMARY KEY ("PostID")
);

CREATE INDEX ON "Post" ("UserName");
\end{verbatim}

\caption{
 Szablon generacji definicji tabel i~przykładowy wynik jego działania.
 Metoda $FormatPrimaryKey$ klasy $CqlHelper$ formatuje kolumny klucza głównego.
}
\label{fig:single:template_table}
\end{figure}


\ref{fig:single:template_class}
\begin{figure}[!ht]
Szablon:

\begin{verbatim}
<#@ template language="C#" #>
<#@ parameter type="View" name="Metadata" #>
public class <#= Metadata.Name #> : IView
{
<# foreach (var field in Metadata.Fields) { #>
 public <#= TypesMap.GetDotNetType(field.Type, field.IsNullable) #>
  <#= field.Name #> { get; set; }
<# } #>
}
\end{verbatim}

Wynik dla widoku $Post$:

\begin{verbatim}
public class Post : IView
{
 public Guid PostID { get; set; }
 public string Title { get; set; }
 public string Content { get; set; }
 public int CommentsNumber { get; set; }
 public string UserName { get; set; }
}
\end{verbatim}

\caption{
 Szablon generacji klas i~przykładowy wynik jego działania.
 Metoda $GetDotNetType$ klasy $TypesMap$ mapuje typ kolumny CQL na typ właściwości C\#.}
\label{fig:single:template_class}
\end{figure}


\ref{fig:single:template_docs}
\begin{figure}[!ht]
Szablon:

\begin{verbatim}
<#@ template language="C#" #>
<#@ parameter type="IEnumerable<Schema.Model.Entity>" name="Metadata" #>
<html>
<# foreach (var entity in Metadata) { #>
 <div>
  <#= entity.Name #>:
  <ul>
<# foreach (var field in entity.Fields) { #>
   <li><#= field.Name #> (<#= field.Type #>)</li>
<# } #>
  </ul>
 </div>
<# } #>
}
</html>
\end{verbatim}

Wynik:

\begin{verbatim}
<html>
 <div>
  Post:
  <ul>
   <li>PostID (timeuuid)</li>
   <li>Title (string)</li>
   <li>Content (string)</li>
   ...
  </ul>
 </div>
 ...
</html>
\end{verbatim}

\caption{Szablon generacji dokumentacji i~fragment wyniku jego działania.}
\label{fig:single:template_docs}
\end{figure}




%=======
\section{Implementacja kolejnych modułów aplikacji}
%=======

\begin{itemize}
 \item CQL (create keyspace, create tables (do pojedynczego skryptu inicjalizującego bazę danych)
 \item DAL (repozytorium bazowe zahardkodowane, konkretnych nie da się wygenerować)
 \item BLL (obsługa zdarzeń zahardkodowana, da się wygenerować Eventy, a EventHandlery?)
 \item WEB (Nancy, da się wygenerować ViewModele, formy)
\end{itemize}
