\chapter{Implementacja generatora} \label{chap:implementation:single_type}

\begin{itemize}
 \item Wybrór klienta Cassandry (?)  
\end{itemize}



%=======
\section{Co będzie generowane, a~co nie}
%=======

Założenia dotyczące całości:

\begin{itemize}
 \item generator ma generować dziedzinę aplikacji CQRS wykorzystująej Cassandrę:
  \begin{itemize}
   \item czy w ogóle generować schemat bazy danych (no schema) (?)
   \item schemat DLL (czyli tak)
   \item klasy C\#: model komend, model zapytań
   \item dokumentancja HTML
  \end{itemize}
 \item ...
\end{itemize}

Ogólnie nie chodzi o to, żeby w ogóle nie pisać nazw klas / właściwości - tylko o to, żeby nie trzeba było pamiętać o wszystkich miejscach, gdzie dana encja jest używana.



%=======
\section{Przykład generowanej aplikacji}
%=======

Sformułowanie przykładu aplikacji

\ref{fig:single:domain_normalized}
\begin{figure}[!ht]
\begin{center}
\begin{tikzpicture} 
\umlclass[x=4,y=5]{User}{ 
 UserName\\
 FirstName\\
 LastName
}{}

\umlclass[y=2]{Like}{ 
 UserName\\
 PostID
}{}

\umlclass[x=4]{Post}{ 
 PostID\\
 Author\\
 Title\\
 Content
}{}

\umlclass[x=9,y=2]{Comment}{ 
 CommentID\\
 PostID\\
 Author\\
 ParentCommentID\\
 Content
}{}

\umluniassoc{Like}{User}
\umluniassoc{Like}{Post}
\umluniassoc{Comment}{User}
\umluniassoc{Comment}{Post}
\umluniassoc{Comment}{Comment}
\umluniassoc{Post}{User}

\end{tikzpicture}
\end{center}

\caption{Znormalizowana dziedzina.}
\label{fig:single:domain_normalized}
\end{figure}


\ref{fig:single:domain_denormalized}
\begin{figure}[!ht]
\begin{center}
\begin{tikzpicture} 
\umlclass[y=4]{User}{ 
 UserName\\
 FirstName\\
 LastName
}{}

\umlclass{UserLike}{ 
 UserName\\
 PostID\\
 PostTitle
}{}

\umlclass[x=4.5,y=4]{Post}{ 
 PostID\\
 Author\\
 Title\\
 Content\\
 CommentsNumber
}{}

\umlclass[x=4.5]{PostLike}{ 
 PostID\\
 UserName\\
 UserFirstName\\
 UserLastName
}{}

\umlclass[x=10,y=2]{Comment}{ 
 CommentID\\
 PostID\\
 Author\\
 ParentCommentID\\
 Content
}{}

\end{tikzpicture}
\end{center}

\caption{Zdenormalizowana dziedzina.}
\label{fig:single:domain_denormalized}
\end{figure}



%=======
\section{Sposób definicji dziedziny}
%=======

\begin{itemize}
 \item że będzie schemat opisu \ref{sec:core:domain_schema_requirement}
 \item ale że schema powinna być w jednym miejsu - będzie jako klasa w kodzie
 \item zdefiniowanie schematu opisu dziedziny na potrzeby CQRS (zapisać każdy w JSONie)
  \begin{itemize}
   \item każda tabela osobno (duplikacje)
   \item PresentIn
  \end{itemize}
 \item które typy umieścić w definicji? CQLowe, .NETowe, czy własne? (nie ma potrzeby własnych, bo trzeba by robić dwie mapy; lepiej CQLowe bo jeden .NETowy mapuje się na wiele CQLowych) (że będzie potrzebna TypesMap.GetDotNetType)
 \item organizacja plików wejściowych i wyjściowych \ref{sec:core:files_structure}
\end{itemize}

\ref{fig:single:model_presentIn}
\begin{figure}[!ht]
\begin{verbatim}
{
 "Name": "User",
 "IsView": true
 "Fields": [
  {
   "Name": "UserName",
   "Type": "text",
   "IsKey": true,
   "PresentInViews": [
    { "Name": "PostLike", "IsKey": true },
    { "Name": "UserLike", "IsKey": true },
    { "Name": "Post" },
    { "Name": "Comment" }
   ],
   "PresentInEvents": [
    "PostPublishedEvent",
    "PostLikedEvent",
    "PostCommentedEvent"
   ]
  },
  {
   "Name": "FirstName",
   "Type": "text",
   "PresentInViews": [ { "Name": "UserLike" } ]
  },
  {
   "Name": "LastName",
   "Type": "text",
   "PresentInViews": [ { "Name": "UserLike" } ]
  }
 ]
}
\end{verbatim}
\caption{
 Pełny opis encji $User$ w~notacji, w~której za pojedyczny element dziedziny przyjęto encję.
 $IsView$ - wartość oznaczająca, że wszystkie pola encji obecne są w~widoku $User$ (likwiduje to potrzebę wymieniania tego widoku w~weźle $PresentInViews$ każdego pola).
}
\label{fig:single:model_presentIn}
\end{figure}


\ref{fig:single:model_perView}
\begin{figure}[!ht]
\begin{verbatim}
{
 "Name": "Post",
 "Type": "View",
 "Fields": [
  {
   "Name": "PostID",
   "Type": "timeuuid",
   "IsKey": true
  },
  {
   "Name": "Title",
   "Type": "text"
  },
  {
   "Name": "Content",
   "Type": "text"
  },
  {
   "Name": "CommentsNumber",
   "Type": "int"
  }
 ]
}

{
 "Name": "Comment",
 "Type": "View",
 "Fields": [
  {
   "Name": "PostID",
   "Type": "timeuuid"
  },
  ...
 ]
}

{
 "Name": "PostCommentedEvent",
 "Type": "Event",
 "Fields": [
  {
   "Name": "PostID",
   "Type": "timeuuid"
  },
  ...
 ]
}
\end{verbatim}
\caption{Opis encji $Post$ w~notacji ``per view''.}
\label{fig:single:model_perView}
\end{figure}



%=======
\section{Podstawowe jednostki generacji}
%=======

\begin{itemize}
 \item entity
 \item event
 \item view
\end{itemize}



%=======
\section{Implementacja szablonów generacji}
%=======

Implementacja szablonów

\begin{itemize}
 \item CQL:
  \begin{itemize}
   \item create table
   \item select
  \end{itemize}
 \item C\#:
  \begin{itemize}
   \item entity
   \item event
   \item view
  \end{itemize}
 \item HTML (dokumentancja)
\end{itemize}

\ref{fig:single:template_table}
\begin{figure}[!ht]
Szablon:

\begin{verbatim}
<#@ template language="C#" #>
<#@ parameter type="Schema.Model.View" name="Metadata" #>
CREATE TABLE "<#= Metadata.Name #>" (
<# foreach (var field in Metadata.Fields) { #>
 "<#= field.Name #>" <#= field.Type #>,
<# } #>
 <#= CqlHelper.FormatPrimaryKey(Metadata) #>
);

<# foreach (var field in Metadata.Fields.Where(x => x.IsSearchable)) { #>
CREATE INDEX ON "<#= Metadata.Name #>" ("<#= field.Name #>");
<# } #>
\end{verbatim}

Wynik dla widoku $Post$:

\begin{verbatim}
CREATE TABLE "Post" (
 "PostID" timeuuid,
 "Title" text,
 "Content" text,
 "CommentsNumber" int,
 "UserName" text,
 PRIMARY KEY ("PostID")
);

CREATE INDEX ON "Post" ("UserName");
\end{verbatim}

\caption{
 Szablon generacji definicji tabel i~przykładowy wynik jego działania.
 Metoda $FormatPrimaryKey$ klasy $CqlHelper$ formatuje kolumny klucza głównego.
}
\label{fig:single:template_table}
\end{figure}


\ref{fig:single:template_class}
\begin{figure}[!ht]
Szablon:

\begin{verbatim}
<#@ template language="C#" #>
<#@ parameter type="View" name="Metadata" #>
public class <#= Metadata.Name #> : IView
{
<# foreach (var field in Metadata.Fields) { #>
 public <#= TypesMap.GetDotNetType(field.Type, field.IsNullable) #>
  <#= field.Name #> { get; set; }
<# } #>
}
\end{verbatim}

Wynik dla widoku $Post$:

\begin{verbatim}
public class Post : IView
{
 public Guid PostID { get; set; }
 public string Title { get; set; }
 public string Content { get; set; }
 public int CommentsNumber { get; set; }
 public string UserName { get; set; }
}
\end{verbatim}

\caption{
 Szablon generacji klas i~przykładowy wynik jego działania.
 $GetDotNetType$ - mapuje typ języka CQL na typ języka C\# (patrz: sekcja~\ref{sec:field_type_definition}).}
\label{fig:single:template_class}
\end{figure}


\ref{fig:single:template_docs}
\begin{figure}[!ht]
Szablon:

\begin{verbatim}
<#@ template language="C#" #>
<#@ parameter type="View[]" name="Metadata" #>
<html>
<# foreach (var entity in Metadata) { #>
 <div>
  <#= entity.Name #>:
  <ul>
<# foreach (var field in entity.Fields) { #>
   <li><#= field.Name #> (<#= field.Type #>)</li>
<# } #>
  </ul>
 </div>
<# } #>
</html>
\end{verbatim}

Wynik:

\begin{verbatim}
<html>
 <div>
  Post:
  <ul>
   <li>PostID (timeuuid)</li>
   <li>Title (text)</li>
   ...
  </ul>
 </div>
 ...
</html>
\end{verbatim}

\caption{Szablon generacji dokumentacji i~fragment wyniku jego działania.}
\label{fig:single:template_docs}
\end{figure}




%=======
\section{Implementacja kolejnych modułów aplikacji}
%=======

\begin{itemize}
 \item CQL (create keyspace, create tables (do pojedynczego skryptu inicjalizującego bazę danych)
 \item DAL (repozytorium bazowe zahardkodowane, konkretnych nie da się wygenerować)
 \item BLL (obsługa zdarzeń zahardkodowana, da się wygenerować Eventy, a EventHandlery?)
 \item WEB (Nancy, da się wygenerować ViewModele, formy)
\end{itemize}
