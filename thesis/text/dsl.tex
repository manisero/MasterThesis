\chapter{DSL}

- że ma być czytelne dla człowieka
- że jak najbardziej business-friendly
- że wynik taki sam jak dla opisu w poprzednim rozdziale
- różnica tylko w opisie dziedziny i algorytmie wyodrębniania jednostek generacji (szablony i wszystko inne takie samo)

%=======
\section{Składnia}
%=======

- że typy własne (id zmiast timeuuid), żeby było zrozumiałe


\subsection{Encje i~ich pola}

\begin{itemize}
 \item nazwy encji poprzedzone są znakiem małpy:
  \begin{itemize}
   \item przykład: \verb|@User|;
  \end{itemize}
  
 \item nazwy pól wchodzących w~skład klucza encji wymienione są w~jej identyfikatorze, w~nawiasach okrągłych bezpośrednio po nazwie encji:
  \begin{itemize}
   \item przykład: \verb|@User(UserName)|,
   \item nazwy pól wchodzących w~skład klucza złożonego oddzielone są przecinkami (bez spacji),
   \item klucz encji wystarczy zdefiniować w~jednym (dowolnym) wystąpieniu jej identyfikatora,
   \item w~opisie dziedziny nie mogą wystąpić dwie różne definicje klucza tej samej encji,
   \item definicja klucza encji musi wystąpić przy co najmniej jednym wystąpieniu jej identyfikatora;
  \end{itemize}
 
 \item pola należące do encji występują po jej identyfikatorze, oddzielone od niego znakiem dwukropka:
  \begin{itemize}
   \item przykład: \verb|@User:FirstName|;
  \end{itemize}
  
 \item typ pola podany jest nawiasach kwadratowych bezpośrednio po nazwie pola
  \begin{itemize}
   \item przykłady: \verb|@Post(PostID[id])|, \verb|@Post:CommentsNumber[int]|,
   \item typ pola wystarczy zdefiniować przy jednym (dowolnym) wystąpieniu jego nazwy,
   \item w~opisie dziedziny nie mogą wystąpić dwie różne definicje typu tego samego pola,
   \item polom, których definicja typu nie występuje ani razu, nadawany jest typ $text$;
  \end{itemize}
 
 \item relację ``wiele do jednego'' pomiędzy encjami oznacza się znakiem dwukropka, oddzielającym identyfikatory tych encji:
  \begin{itemize}
   \item przykład: \verb|@Post:@User:FirstName|,
   \item znaczenie: \emph{z~jedną instancją encji} \verb|@User| \emph{powiązanych jest wiele instancji encji} \verb|@Post|;
  \end{itemize}
 
 \item relację ``jeden do wielu'' pomiędzy encjami oznacza się dwoma znakami dwukropka, oddzielającymi identyfikatory tych encji:
  \begin{itemize}
   \item przykład: \verb|@Post::@Comment:Content|,
   \item znaczenie: \emph{z~jedną instancją encji} \verb|@Post| \emph{powiązanych jest wiele instancji encji} \verb|@Comment|.
  \end{itemize}
\end{itemize}


\subsection{Zdarzenia}

\begin{itemize}
 \item sekcja definicji zdarzeń następuje w~opisie dziedziny po linii o~treści ``\verb|Events:|'';
 \item definicja pojedynczego zdarzenia składa się z dwóch części:
  \begin{itemize}
   \item nazwy zdarzenia,
   \item opisu zdarzenia, zawierającego identyfikatory i~pola encji wchodzących w~jego skład;
  \end{itemize}
 \item przykład: \verb|PostPublished: @Post with @Post:Content can be published|;
 \item definicje kolejnych zdarzeń są od siebie oddzielone pustymi liniami.
\end{itemize}

Pojedyncza definicja zdarzenia jednoznacznie określa pojedyncze zdarzenie wchodzące w~skład modelu komend systemu.


\subsection{Widoki}

\begin{itemize}
 \item sekcja definicji widoków następuje w~opisie dziedziny po linii o~treści ``\verb|Views:|'';
 \item definicja widoku to pojedyncze zdanie zawierające identyfikatory i~pola encji w~nim występujących;
 \item przykład: \verb|I can display @Post with @Post:Title and @Post:Content|;
 \item definicje kolejnych widoków są od siebie oddzielone pustymi liniami.
\end{itemize}

Pojedyncza definicja widoku nie określa jednoznacznie pojedynczego widoku wchodzącego w~skład modelu zapytań systemu.
Określa natomiast treść wyświetlaną na pojedynczym ekranie aplikajcji, co z~biznesowego punktu widzenia jest bardziej intuicyjne.

Model zapytań będzie więc generowany stopniowo na podstawie analizy wszystkich definicji widoków zawartych w~opisie dziedziny.
Po analizie pojedynczej definicji widoku będzie następować aktualizacja całej definicji modelu zapytań.



%=======
\section{Przykład}
%=======

(patrz:~\ref{sec:sample_app})

przedstawia rysunek~\ref{fig:dsl:model_dsl}
\begin{figure}[!ht]
\begin{verbatim}
Events:

PostPublished:
@User(UserName) of @User:UserName can publish a @Post(PostID[id]) with
@Post:Title and @Post:Content.

PostLiked:
@User of @User:UserName can like a @Post by its @Post:PostID.

PostCommented:
@User of @User:UserName can write a @Comment(CommentID[id]) of
@Comment:Content under a post of ID @Comment:PostID or another comment
of ID @Comment:ParentCommentID[id].

========

Views:

I can display @Post list with @Post:@User:UserName, @Post:Title,
@Post:Content and @Post:CommentsNumber[int].

I can display @Post details with @Post:@User:UserName, @Post:Title,
@Post:Content, @Post::@Comment:Content, @Post::@Comment:@User:UserName,
@Post::@User:FirstName and @Post::@User:LastName.

I can display user profile with @User:UserName, @User:FirstName,
@User:LastName and @User::@Post:Title.

\end{verbatim}
\caption{
 Pełny opis dziedziny systemu w~języku DSL.
}
\label{fig:single:model_dsl}
\end{figure}


Należy zwrócić uwagę na czytelność i~zwięzłość tego opisu.
Żaden z~przedstawionych wcześniej sposobów opisu dziedziny nie mógł pokazany w~całości na jednej stronie, podczas gdy powyższy opis mieści się bez problemu.



%=======
\section{Algorytm}
%=======

\begin{enumerate}
 \item Stwórz pustą listę widoków $V$ i~pustą listę zdarzeń $E$.
 \item Dla każdej definicji zdarzenia zawartej w~sekcji definicji zdarzeń:
  \begin{enumerate}
   \item todo
  \end{enumerate}
 \item Dla każdej definicji widoku zawartej w~sekcji definicji widoków:
  \begin{enumerate}
   \item todo
  \end{enumerate}

\end{enumerate}



%=======
\section{Kroki postępowania}
%=======



%=======
\section{Podsumowanie}
%=======

- że wystarczający na potrzeby tego systemu
- że gdyby pojedyncze pole miało więcej właściwości, to lipa
- że bardzo business-friendly
