\chapter{Generacja kodu} \label{chap:code_generation}

W~każdej aplikacji obiektowej korzystającej z bazy danych występuje duplikacja opisu dziedziny aplikacji.
Występuje ona w~co najmniej dwóch miejsach:

\begin{enumerate}
 \item schemacie bazy danych (DDL),
 \item definicjach klas w~kodzie źródłowym aplikacji.
\end{enumerate}

W~miarę jak rozrasta się projekt informatyczny, pojawia się tendencja do duplikowania fragmentów dziedziny poruszanego przez niego problemu.
Duplikacja ta rozprzestrzenia się pomiędzy modułami aplikacji, których zadaniem obróbka tych samych danych, ale w~różny sposób.
Przykładowo, aplikacja może udostępniać przechowywane dane na następujące sposoby:

\begin{itemize}
 \item wyświetlać je na stronie WWW,
 \item wystawiać jako API,
 \item eksportować do arkusza kalkulacyjnego.
\end{itemize}

Jeśli moduły te posiadają osobne implementacje dziedziny aplikacji, to każda zmiana dziedziny wymaga zmodyfikowania implementacji dziedziny w~każdym z~modułów - co wiąże się z~dużymi kosztami.

Z~drugiej strony, identyczność zestawu danych udostępnianego przez różne moduły może nie być pożądana.
Przykładowo, na stronie WWW wyświetlane mogą być jedynie podstawowe dane danej encji, podczas gdy pełne dane dostępne są po wyeksportowaniu arkusza kalkulacyjnego.
Takie wymaganie wymusza duplikację części dziedziny aplikacji pomiędzy różnymi implementacjami tej dziedziny.

Rozwiązaniem problemu może być osiągnięcie implementacji, w~której pełna dziedzina aplikacji definiowana jest w jednym miejscu, a~jej implementacje są generowane automatycznie w~odpowiednich modułach.
Taka sytuacja sprawia, że duplikacja pomiędzy modułami przestaje być problemem - aby wprowadzić zmiany we wszystkich implementacjach, wystarczy wprowadzic pojedynczą modyfikację w~definicji dziedziny aplikacji.

Najprostszym przykładem realizującym to rozwiązanie jest użycie generatora definicji klas na podstawie schematu bazy danych (tzw. podejście database-first) lub generatora schematu bazy danych na podstawie definicji klas (tzw. podejście code-first).
