\chapter{Duplikacja} \label{chap:duplication}

Zasada ``DRY'' (ang. \emph{``Don't Repeat Yourself''}), sformułowana przez Andrew Hunt'a i~David'a Thomas'a, mówi: ``Każda porcja wiedzy powinna mieć pojedynczą, jednoznaczną i autorytatywną reprezentację w~systemie''~\cite{prag_prog:dupl}.
Poprawne jej stosowanie skutkuje osiągnięciem stanu, w~którym pojedyncza zmiana zachowania systemu wymaga modyfikacji tylko jednego fragmentu reprezentacji wiedzy.
Należy podkreślić, że autorzy tej zasady przez ``reprezentację wiedzy'' rozumieją nie tylko kod źródłowy tworzony przez programistów systemu.
Zaliczają do niej również dokumentację systemu, strukturę bazy danych przez niego używanej oraz inne artefakty powstające w~procesie wytwarzania oprogramowania (np. scenariusze testów akceptacyjnych), a~nawet czynności wykonywane rutynowo przez programistów.

Celem stosowania zasady ``DRY'' jest unikanie duplikacji wiedzy zawartej w~systemie.
Duplikacją określa się fakt występowania w~systemie dwóch reprezentacji tej samej porcji wiedzy.
Należy wyjaśnić, że wielokrotne występowanie nazw klas lub zmiennych (bądź innych identyfikatorów) w~kodzie programu nie jest duplikacją - identyfikatory nie są reprezentacją więdzy, a~jedynie odnośnikami do wiedzy reprezentowanej przez artefakty, na które wskazują.



%=======
\section{Rodzaje duplikacji} \label{sec:dupl_kinds}
%=======

Wyróżnia się cztery rodzaje duplikacji ze względu na przyczynę jej powstania~\cite{prag_prog:dupl}:

\begin{itemize}
 \item duplikacja wymuszona (ang. \emph{imposed duplication}) - pojawia się, gdy programista świadomie duplikuje kod aplikacji (lub inną reprezentację wiedzy) uznając, że w~danej sytuacji nie jest możliwe uniknięcie duplikacji;
 \item duplikacja niezamierzona (ang. \emph{inadvertent duplication}) - występuje wtedy, gdy programista nie jest świadomy, że jego działania prowadzą do powstania duplikacji;
 \item duplikacja niecierpliwa (ang. \emph{impatient duplication}) - jest wynikiem niedbalstwa programisty; ma miejsce w sytuacji, gdy programista, napotkawszy problem, świadomie wybierze rozwiązanie najprostsze, ale prowadzące do powstania duplikacji;
 \item duplikacja pomiędzy programistami (ang. \emph{interdeveloper duplication}) - pojawia się, gdy kilku programistów tworzących tę samą aplikację wzajemnie duplikuje tworzony kod lub tworzy niezależne reprezentacje tej samej wiedzy.
\end{itemize}

Duplikacja może występować w~różnych postaciach:


\subsection{Duplikacja czynności wykonywanych przez programistów} \label{sec:tasks_dupl}

Do rutynowych czynności wykonywanych podczas pracy programisty należą wszelkiego rodzaju aktualizacje: struktury bazy danych rezydującej na maszynie programisty czy wersji tworzonego oprogramowania zainstalowanej w~środowisku testowym lub produkcyjnym.

Czynności te zwykle składają się z~kilku kroków. Przykładowo, na aktualizację bazy danych może składać się:

\begin{enumerate}
 \item Usunięcie istniejącej bazy.
 \item Stworzenie nowej bazy.
 \item Zainicjalizowanie struktury nowej bazy przy pomocy zaktualizowanego skryptu DLL (\emph{Data Definition Language}).
 \item Wypełnienie nowej bazy danymi przy pomocy zaktualizowanego skryptu DML (\emph{Data Manipulation Language}).
\end{enumerate}

Jeśli wszystkie z~tych kroków wykonywane są ręcznie, to mamy do czynienia z~duplikacją - każdy programista w~zespole, zamiast tylko odwoływać się do wiedzy o~tym, jak zaktualizować bazę danych, musi posiadać tę więdzę i~wielokrotnie wcielać ją w~życie krok po kroku.


\subsection{Duplikacja kodu źródłowego aplikacji} \label{sec:code_dupl_kinds}

Programistom niejednokrotnie zdarza się spowodować duplikację w~kodzie źródłowym tworzonych programów.
Powielone fragmenty kodu są zazwyczaj niewielkie, a podzielić je można na następujące kategorie~\cite{soft_sol_dupl}:

\begin{itemize}
 \item prosta duplikacja wyrażeń (ang. \emph{basic literal duplication}) - najprostszy rodzaj duplikacji; obejmuje fragmenty kodu, których treść jest identyczna (przykład: dwie identyczne metody umiejscowione w~różnych modułach systemu);
 \item parametryczna duplikacja wyrażeń (ang. \emph{parametric literal duplication}) - obejmuje fragmenty kodu, których treść różni się jedynie typami danych, na których te fragmenty operują;
 \item duplikacja strukturalna (ang. \emph{structural duplication}) - obejmuje fragmenty kodu, które mają ten sam schemat działania, ale różnią się pojedynczymi instrukcjami (przykład: dwie pętle iterurąjce po tej samej kolekcji z~tym samym warunkiem stopu, ale wykonujące różne operacje na elementach kolekcji);
 \item duplikacja czasowa (ang. \emph{temporal duplication}) - określa fragment kodu, który jest niepotrzebnie wykonynway wiele razy (przykład: zliczenie elementów kolekcji podczas każdego sprawdzenia warunku stopu pętli iterurąjcej po tej kolekcji);
 \item duplikacja intencji (ang. \emph{duplication of intent}) - występuje, gdy dwóch programistów umieści w~różnych modułach aplikacji dwa fragmenty, których wynik działania (ale niekoniecznie treść) jest identyczny (jest to jeden z~przypadków duplikacji pomiędzy programistami).
\end{itemize}


\subsection{Duplikacja opisu dziedziny aplikacji}

W~każdej aplikacji obiektowej korzystającej z~bazy danych opis dziedziny aplikacji występuje w~co najmniej dwóch miejscach.
Te miejsca to:

\begin{itemize}
 \item struktura bazy danych (DDL);
 \item definicje klas w~kodzie źródłowym aplikacji.
\end{itemize}

Kolejnym typowym miejscem, w~którym umieszcza się informacje o~dziedzinie aplikacji jest jej dokumentacja.

W~miarę rozrastania się systemu i~powstawania kolejnych funkcjonalności i~komponentów, pojawia się tendencja do powielania całości bądź części definicji dziedziny w~różnych modułach.
Dotyczy to głównie tych modułów, których zadaniem jest obróbka tych samych danych, ale w~różny sposób.
Przykładowo, aplikacja może pozwalać na dostęp do przechowywanych danych na następujące sposoby:

\begin{itemize}
 \item wyświetlać je na stronach WWW,
 \item udostępniać je poprzez usługi sieciowe jako API (\emph{Application Programming Interface}),
 \item umożliwiać ich eksport do arkusza kalkulacyjnego.
\end{itemize}

Komponenty realizujące te funkcjonalności mogą korzystać z~pojedynczej implementacji dziedziny.
Może to jednak nie być pożądane, szczególnie w~przypadku, gdy każdy ze sposobów udostępnia inny zestaw danych.
Przykładowo, strony WWW mogą wyświetlać podstawowe dane każdej encji będącej częścią dziedziny, podczas gdy API udostępnia pełne dane wszystkich encji, a eksport do arkusza kalkulacyjnego udostępnia pełne dane tylko niektórych encji.

W~takim przypadku komponent odpowiedzialny za API operuje na pełnej dziedzinie, a~komponenty stron WWW i~eksportu do arkusza kalkulacyjnego - jedynie na jej fragmentach.
Jeśli każdy z~tych komponentów posiada niezależną implementację dziedziny aplikacji, to mamy do czynienia z~duplikacją.



%=======
\section{Skutki występowania duplikacji}
%=======

Pojawienie się duplikacji w~systemie ma zazwyczaj szkodliwe skutki.
Rozprzestrzenia się ona tym szybciej, a wyrządzane szkody są tym dotkliwsze, im więszy jest rozmiar systemu.

Najbardziej oczywistym skutkiem występowania duplikacji jest wydłużenie się czasu poświęcanego przez programistów na nieskomplikowane lub powtarzalne zadania.
Przykładowo, nieznajomość kodu współdzielonego przez wszystkie komponenty systemu (tzw. rdzenia, ang. \emph{core}) powoduje, że programiści niepotrzebnie spędzają czas na implementowaniu podstawowych funcji, które są już dostępne w~rdzeniu systemu.
Innym przykładem może być praktyka ręcznego wykonywania czynności, które mogą być wykonywane automatycznie, w~tle.

Po drugie, pojedyncza porcja wiedzy jest reprezentowana w~wielu miejsach, a~więc zmiana pojedynczego zachowania systemu wymaga modyfikacji wielu jego modułów.
Wprowadzając dowolną modyfikację, programista musi ręcznie zlokalizować wszystkie miejsca wymagające zaktualizowania.
Problem ten jest najbardziej odczuwalny podczas naprawy zgłoszonych błędów: będąc świadomym występowania duplikacji kodu, programista naprawiający błąd musi pamiętać, że poprawiany przez niego fragment może być powielony w~wielu miejscach w~systemie.
Wszystkie te miejsca musi zlokalizować i~poprawić~\cite{repetition}.
Jeśli tego nie zrobi, w~systemie pozostaną sprzeczne ze sobą reprezentacje danej wiedzy, a~kolejni programiści nie będą świadomi, która reprezentacja jest poprawna.
W dłuższej perspektywie taka sytuacja prowadzi do pojawienia się kolejnych błędów.

Po trzecie, duplikacja powoduje zbędne powiększenie rozmiarów systemu.
Kod, w~którym występują zduplikowane fragmenty jest dłuższy, a przez to mniej czytelny.
Co więcej, zwiększona ilość kodu powoduje niepożądane zwiększenie rozmiarów plików wchodzących w~skład systemu.

Należy również zwrócić uwagę na zagrożenie, jakie niesie ze sobą wystąpienie pojedynczej duplikacji, której istnienie zostanie zaaprobowane przez programistów wchodzących w~skład zespołu tworzącego projekt.
Może to spowodować osłabienie dyscypliny w~zespole, a~w~efekcie doprowadzić do pojawienia się kolejnych duplikacji, a~także innych złych praktyk programistycznych.
Pojedyncze wystąpienie tego zjawiska może więc doprowadzić do rozprzestrzenia się innych niepożądanych zjawisk i~pogorszenia jakości całego systemu~\cite{prag_prog:entropy}.



%=======
\section{Rozwiązanie: utrzymywanie listy zduplikowanych fragmentów}
%=======

Najprostszym, a~zarazem najbardziej naiwnym sposobem na zniwelowanie skutków duplikacji może być utrzymywanie listy fragmentów systemu, w~których duplikacja występuje.
Przykładowo, zespół może przygotować listę miejsc, które należy zmodyfikować, jeśli pojedyncza encja zostanie wzbogacona o~nowe pole.
Na takiej liście znalazłaby się struktura bazy danych, implementacja modelu danych w~kodzie źródłowym programu, dokumetacja systemu, itd.

Takie podejście ma jednak istotne wady:

\begin{itemize}
 \item rozwiązanie to nie powstrzymuje programistów przed wprowadzaniem kolejnych duplikacji, a~wręcz na to przyzwala;
 \item pojawienie się nowego miejsca występowania pól encji (np. alternatywnej implementacji modelu danych) wymaga zaktualizowania listy; niezaktualizowana lista może skutkować pominięciem nowych miejsc w~szacunkach dotyczących czasu potrzebnego na implementację nowych funkcjonalności i~w~samej implementacji tych funkcjonalności;
 \item rozwiązanie to niweluje tylko drugi z~wymienionych skutków występowania duplikacji - programista nie musi samodzielnie szukać miejsc, w~których powinien wprowadzić modyfikacje.
\end{itemize}



%=======
\section{Rozwiązanie: skrypty automatyzujące rutynowe czynności} \label{sec:tasks_scripts}
%=======

Wiele czynności wykonywanych przez programistów da się zautomatyzować, tworząc skrypty je wykonujące.
Przykłady:

\begin{itemize}
 \item wszystkie kroki składające się na czynność aktualizacji shcematu bazy przedstawone w sekcji~\ref{sec:tasks_dupl} mogą być wykonywane przez pojedynczy skrypt powłoki korzystający z~kilku skryptów DDL i DML;
 \item operacja instalacji nowej wersji systemu na środowisku testowym lub produkcyjnym nie musi wiązać się z~ręcznym przesyłaniem plików systemu na serwer FPT zdalnej maszyny - może odbywać się automatycznie, po uruchomieniu odpowiedniego skryptu lub zarejestrowaniu zmiany kodu źródłowego w~systemie kontroli wersji~\cite{cont_delivery}.
\end{itemize}

Taka automatyzacja ma wiele zalet, w~tym:

\begin{itemize}
 \item w~trakcie działania skryptu w tle, programista może skupić się na innych zadaniach,
 \item wiedza na temat kroków składających się na daną czynność jest reprezentowana w~jednym miejscu (skrypcie),
 \item programiści lub inne skrypty odwołują się do tej wiedzy poprzez jej identyfikator (nazwę lub ścieżkę w~systemie plików).
\end{itemize}

Należy jednak zauważyć, że podejście to rozwiązuje jedynie problem duplikacji czynności wykonywanych przez programistów.
Pozostałe postacie duplikacji nie mogą być zniwelowanie w~ten sposób.



%=======
\section{Rozwiązanie: generyczna implementacja}
%=======

Rozwiązaniem problemu duplikacji kodu źródłowego aplikacji może być generyczna implementacja aplikacji.
W~językach obiektowych jest ona możliwa do osiągnięcia na kilka sposobów:


\subsection{Dziedziczenie}

Podstawowym sposobem na uniknięcie duplikacji w~kodzie aplikacji wykorzystujących języki obiektowe jest dziedziczene.
Metody wspólne dla rodzin klas obiektów są umieszczane w~ich klasach bazowych.
Pozwala to uniknąć podstawowej duplikacji wyrażeń (patrz: sekcja~\ref{sec:code_dupl_kinds}).


\subsection{Typy szablonowe i~generyczne}

Innych rodzajów duplikacji - parametrycznej duplikacji wyrażeń i~duplikacji strukturalnej - można uniknąć poprzez zastosowanie typów generycznych i~szablonów klas.
Kod o~tej samej strukturze i~działaniu, ale operujący na danych innego typu, może być zamknięty w~pojedynczej klasie generycznej.


\subsection{Refleksja}

W~sytuacjach, w~których występuje potrzeba zaimplementowania podobnej funkcjonalności dla kilku typów obiektów, które nie należą do jednej rodziny, można skorzystać z~mechanizmu refleksji.
Pozwala on osiągnąć rezultaty podobne do tych oferowanych przez klasy bazowe i~typy generyczne.\\


Należy podkreślić, że przedstawione propozycje rozwiązują jedynie problem duplikacji kodu źródłowego aplikacji.


%=======
\section{Rozwiązanie: użycie generatorów}
%=======

Rozwiązaniem pozwalającym na uniknięcie duplikacji opisu dziedziny aplikacji jest zastosowanie różnego typu generatów.

Należy wyjaśnić, że kodu źródłowego (bądź innych artefaktów) wygenerowanego na podstawie innego kodu nie uznaje się za duplikację.
Powodem jest to, że jeśli na podstawie danego fragmentu kodu generowanych jest wiele artefaktów, to aby wprowadzić zmiany we wszystkich tych artefaktach, należy jedynie  zmodyfikować źródłowy fragment kodu i~uruchomić proces generacji.
Za ``pojedynczą, jednoznaczną i autorytatywną'' reprezentację danej porcji wiedzy (patrz: Zasada ``DRY'') uznaje się w tym przypadku źródłowy fragment kodu.


\subsection{Generator modelu dziedziny aplikacji na podstawie struktury bazy danych} \label{sec:database_first}

Najpowszechniejszym przykładem użycia generatorów kodu jest wykorzystanie generatora zestawu klas wchodzących w~skład modelu dziedziny aplikacji na podstawie struktury bazy danych używanej przez tę aplikację.
Takie podejście nosi nazwę ``najpierw baza danych'' (ang. \emph{Database First}).

Przykładami narzędzi umożliwiających taką generację kodu są:

\begin{itemize}
 \item EntityFramework\footnote{EntityFramework - strona projektu: https://entityframework.codeplex.com/} - rozbudowane narzędzie ORM (ang. \emph{Object-Relational Mapping}) przeznaczone na platformę .Net;
 \item SQLMetal\footnote{SQLMetal - strona projektu: http://msdn.microsoft.com/pl-pl/library/bb386987(v=vs.110).aspx} - narzędzie dla platformy .Net, którego jedynym zadaniem jest generacja kodu źródłowego klas na podstawie struktury bazy danych;
 \item Django\footnote{Django - strona projektu: https://www.djangoproject.com/} - platforma aplikacji webowych dla języka Python.
\end{itemize}


\subsection{Generator struktury bazy danych na podstawie modelu dziedziny aplikacji}

Podejściem przeciwstawnym dla \emph{``Database First''} jest podejście ``najpierw kod'' (ang. \emph{Code First}).
Jak sama nazwa wskazuje, generatory tego typu generują strukturę bazy danych na podstawie klas należących do implementacji modelu w~kodzie źródłowym aplikacji.

Przykłady generatorów \emph{``Code First''} to:

\begin{itemize}
 \item Hibernate\footnote{Hibernate - strona projektu: http://hibernate.org/} - narzędzie ORM dla platform Java i~.Net;
 \item EntityFramework (patrz: sekcja~\ref{sec:database_first});
 \item Django (patrz: sekcja~\ref{sec:database_first}).
\end{itemize}


\subsection{Generator dokumentacji}

Duplikacji opisu dziedziny aplikacji w~jej dokumentacji można uniknąć poprzez zastosowanie narzędzi generujących tę dokumentację na podstawie kodu źródłowego aplikacji.
Generatory tego rodzaju wymagają umieszczania w~kodzie źródłowym specjalnie sformatowanych komentarzy, na podstawie których są w~stanie wygenerować dokumentację w~kilku formatach, takich jak PDF czy HTML.
Przykłady takich narzędzi to:

\begin{itemize}
 \item Doxygen\footnote{Doxygen - strona projektu: http://www.stack.nl/~dimitri/doxygen/} - popularne narzędzie obsługujące wiele języków programowania (w~tym C++, Java C\#), wspierające wiele formatów dokumentacji (w~tym HTML, PDF, LaTeX, XML);
 \item JavaDoc\footnote{JavaDoc - strona projektu: http://www.oracle.com/technetwork/java/javase/documentation/index-jsp-135444.html} - generator dedykowany językowi Java, dymyślnie wspiera jedynie format HTML;
 \item C\# XML Documentation\footnote{C\# XML Documentation - strona projektu: http://msdn.microsoft.com/en-us/library/b2s063f7(vs.\\71).aspx} - format tworzenia dokumentacji wbudowany w~język C\#, na podstawie którego generowana jest dokumentacja w~formacie XML;
 \item pydoc\footnote{pydoc - strona projektu: https://docs.python.org/2/library/pydoc.html} - narzędzie będące częścią standardowego zestawu narzędzi deweloperskich języka Python; wspiera format tekstowy i~HTML.
\end{itemize}


\subsection{Inne generatory}

Istnieje wiele innych generatorów, należących do typu programowania nazywanego programowaniem automatycznym (ang. \emph{automatic programming})~\cite{auto_prog}.
Jednakże większość z~nich, tak jak powyższe przykłady, eliminuje tylko pojedyncze rodzaje duplikacji.\\

Warto zwrócić uwagę na fakt, że różne typy generatorów za źródło danych obierają sobie różne definicje dziedziny: np. generator \emph{Database First} bazuje na strukturze bazy danych, a~generator dokumentacji - na kodzie źródłowym aplikacji.
Używanie wielu różnych generatorów usuwających pojedyncze rozdzaje duplikacji w końcu doprowadziłoby do powstania ``łańcucha'' generatów, w~którym wynik działania jedngo generatora byłby źródłem danych dla innego.
Takie rozwiązanie mogłoby być trudne w~utrzymaniu.



%=======
\section{Rozwiązanie: pojedynczy generator wszystkich artefaktów systemu}
%=======

Przedstawione rozwiązania mają pewne wspólne wady.
Po pierwsze, wszystkie z~nich usuwają tylko pojedyczne rodzaje duplikacji.
Aby całkowicie wyeliminować duplikację z~systemu, należałoby by użyć prawie wszystkich z~nich.

Po drugie, rozwiązania zwalczające duplikację czynności wykonywanych przez programistów i~duplikacje kodu źródłowgo aplikacji nie są rozwiązaniami prewencyjnymi - nie zapobiegają pojawianiu się nowych duplikacji.
To, czy nowa rutynowa czynność zostanie zautomatyzowana przy pomocy skryptu powłoki, i~czy nowy moduł aplikacji zostanie zaimplementowany w~sposób generyczny, zależy jedynie od dyscypliny i~dbałości zespołu tworzącego system.
Aby uniknąć takiej sytuacji, wydaje się, że mechanizmy niwelujące duplikację powinny leżeć u podstawy systemu - tak, aby programiści intuicyjnie korzystali z~nich, wprowadzając nowe czynności lub aktualizując dziedzinę aplikacji.

Odpowiednim rozwiązaniem wydaje się być zastosowanie pojedynczego generatora potrafiącego wygenerować wszystkie potrzebne w~systemie artefakty.
Powstanie takiego generatora dającego się zastosować w~każdym projekcie jest jednak bardzo mało prawdopodobne, ponieważ:

\begin{itemize}
 \item każdy projekt ma inne wymagania odnośnie wygenerowanych artefaktów;
 \item prawdopodobne nie istnieje format, który pozwalałby zdefiniować każdą dziedzinę w~najlepszy dla niej - tj. najbardziej naturalny, a~przy tym pozbawiony duplikacji - sposób.
\end{itemize}



%=======
\section{Podsumowanie}
%======

Jak widać, powszechnie dostępne są jedynie rozwiązania pozwalające wyeliminować pojedyncze rodzaje duplikacji.
Nie jest jednak dostępny pojedynczy, spójny sposób na wyeliminowanie duplikacji w~obrębie całego systemu.
Odpowiednim rozwiązaniem wydaje się być połączenie kilku z~wyżej przedstawionych sposobów.
W~dalszej części pracy podjęta zostanie próba stworzenia takiego rozwiązania.
Zostanie ono oparte o~generację kodu, a~głównym założeniem jest skonstruowanie go w~taki sposób, aby pozwolić na usunięcie jak największej liczby rodzajów duplikacji patrząc zarówno z~perspektywy ich postaci, jak i~przyczyn powstawania~(patrz: sekcja~\ref{sec:dupl_kinds}).

Generacja została wybrana jako rozwiązanie problemu duplikacji dlatego, że pozwala zredukować duplikację nie tylko w~kodzie źródłowym aplikacji, a~w~obrębie całego systemu.
Co więcej, użycie mechanizmów generacji nie przekreśla wykorzystania innych sposobów eliminacji duplikacji.
Przykładowo, skrypty powłoki automatyzujące czynności wykonywane przez programistów również mogą być generowane lub mogą działać na wygenerowanych plikach - wtedy tym łatwiej będzie można je dostosowywać do zmian w~systemie.

\begin{itemize}
\item wniosek: żaden generator nie będzie jednocześnie bogaty w funkcjonalności i dobry dla każdego typu aplikacji
 \item zatem: rozwiązanie podzielę na dwie części
  \begin{itemize}
   \item rdzeń narzędzia do generacji
   \item generator aplikacji jednego typu
  \end{itemize}
\end{itemize}
